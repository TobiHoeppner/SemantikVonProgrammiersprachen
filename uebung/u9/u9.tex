\documentclass[ngerman,a4paper]{report}
\usepackage[ngerman]{babel}
\usepackage[T1]{fontenc}
\usepackage[utf8]{inputenc}
\usepackage{lmodern}
\usepackage{MyriadPro}
\usepackage[scaled]{beramono}
\newcommand\Small{\fontsize{10.5}{10.5}\selectfont}
\newcommand*\LSTfont{\Small\ttfamily\SetTracking{encoding=*}{-20}\lsstyle}
\usepackage{microtype}
\usepackage{geometry}
\geometry{verbose,tmargin=3cm,bmargin=3cm,lmargin=3cm,rmargin=3cm}
\usepackage{centernot}
\usepackage{listings}
\usepackage{paralist}
%\usepackage{array}
\usepackage{xcolor}
%\usepackage{graphicx}
%\usepackage{caption}
%\usepackage{url}
%\usepackage[verification]{struktex}
\usepackage{amsmath}
\usepackage{amsfonts}
\usepackage{mathtools}
%\usepackage{accents}
\usepackage{tikz}
\usetikzlibrary{shapes,arrows,automata}
\tikzset{
	treenode/.style = {align=center, inner sep=1pt, text centered,font=\sffamily},
	node/.style = {treenode, font=\sffamily\bfseries, text width=1.5em},
	cloud/.style = {draw, circle, fill=red!20, node distance=3cm, minimum height=2em},
	decision/.style = {diamond, draw, fill=blue!20, text width=6em, text badly centered, node distance=3cm, inner sep=0pt},
	block/.style = {rectangle, draw, fill=green!20, text width=6em, text centered, rounded corners, minimum height=4em},
	line/.style = {draw, -latex'}
}
\usepackage{stmaryrd}

% Code Listing Style
\definecolor{darkblue}{rgb}{0,0,.6}
\definecolor{darkgreen}{rgb}{0,0.5,0}
\definecolor{darkred}{rgb}{0.5,0,0}
\lstset{%
	language=C,
	basicstyle=\LSTfont,
	commentstyle=\color{darkgreen},
	keywordstyle=\color{darkblue}\bfseries,
	breaklines=true,
	tabsize=2,
	xleftmargin=\fboxsep,
	xrightmargin=-\fboxsep,
	numbers=left,
	numberstyle=\tiny\color{gray},
	stepnumber=1,
	numbersep=5pt,
	frame=bt,
	stringstyle=\color{darkred},
	showstringspaces=false,
	rulecolor= \color{gray},
	emph=[1]%
	{%
	    then, not%
	},
	emphstyle=[1]{\color{darkblue}\bfseries},
	emph=[2]%
	{%  Datatypes
	    %
	},
	emphstyle=[2]{\color{darkblue}\bfseries},
	emph=[3]%
	{%
	    %
	},
	emphstyle=[3]{\color{darkred}\bfseries},
	literate=%
	{Ö}{{\"O}}1
	{Ä}{{\"A}}1
	{Ü}{{\"U}}1
	{ß}{{\ss}}2
	{ü}{{\"u}}1
	{ä}{{\"a}}1
	{ö}{{\"o}}1
}
\providecommand{\tabularnewline}{\\}

\usepackage{fancyhdr}
\pagestyle{fancy}
\usepackage{lastpage}
\makeatletter

\lhead{\textbf{\@title Tutor:} Paul Podlech \\ \@author}
\chead{}
\rhead{\@date \\ \thepage \ von \pageref{LastPage} }
\cfoot{}
%\cfoot{\small \textbf{Disclaimer:} Einige Lösungen wurden mit einer anderen Übungsgruppe (Jens Fischer, Johannes Dillmann, Tobias Famulla) inhaltlich diskutiert,  eine gewisse Ähnlichkeit der Lösungen ist möglich. Trotzdem sind alle Lösungen selbstständig von den hier genannten Mitgliedern erarbeitet.}
\renewcommand{\labelenumi}{\alph{enumi})}
\renewcommand{\maketitle}{}
\newcommand{\utilde}[1]{\underaccent{\tilde}{#1}}
\renewcommand{\familydefault}{\sfdefault}

\author{Florian Ritzel, Hinnerk van Bruinehsen, Tobias Höppner}
\title{Anonyme Semantiker - Übung 09. }
\date{Sommersemester 2014}
\begin{document}
\maketitle
\section*{Aufgabe 1}
Wenn man für die $\alpha$-Reduktion $\lambda x.t \xrightarrow[\alpha]{} \lambda y.\$_y^x\ t$ auf die Bedingung $y \notin \text{\lstinline!Var!}(t)$ verzichtet, kann eine solche Reduktion die Semantik verändern. Geben Sie dafür ein Beispiel an.\\

Die $\alpha$-Konversion funktioniert nur, wenn die Variable noch nicht verwendet wird ($y \notin \text{\lstinline!Var!}(t)$). Wenn man auf diese Eigenschaft verzichtet kann man genauso gut auch gleich das Gegenteil fordern: $y \in \text{\lstinline!Var!}(t)$. Beispiel:
\begin{align*}
\lambda x.xy \xrightarrow[\alpha]{}\lambda y.yy
\end{align*}

\section*{Aufgabe 2}
Wenn man für die $\beta$-Reduktion
\begin{align*}
(\lambda x.t) s \xrightarrow[\beta]{} \$_s^x t
\end{align*}
auf die Forderung $\text{\lstinline!Fr!}(s) \cap \text{\lstinline!Geb!}(t) = \emptyset$ verzichtet, kann eine solche Reduktion die Semantik verändern.\\
Geben Sie dafür ein Beispiel an.\\

Verzichtet man auf die oben genannte Eigenschaft bedeutet das, dass die Schnittmenge der freien Variablen von $s$ und der gebundenen Variablen von $t$ nicht leer ist. Es gibt also eine Variable die in beiden Termen vorkommt und damit nach Anwendung der $\beta$-Reduktion in einem Term stehen:
\begin{align*}
(\lambda x.xyz) z \xrightarrow[\beta]{} \lambda zyz
\end{align*}

\section*{Aufgabe 3}
Konstruieren Sie einen $\lambda$-Ausdruck $t$, der keine Normalform besitzt und dessen Reduktion zu immer größeren Ausdrücken führt.\\
%\textbf{Idee}:
%\begin{align*}
%\lambda x. x (x) \lambda x. x (x) : D \in A_\lambda für\ alle\ x \in \mathcal{X}^D
%\end{align*}
%Sollte eigentlich dazu führen, dass x sich immer wieder selbst aufruft und damit keine Normalform möglich ist.
%
%\textbf{Idee Hinnerk:}
Für das ungetypte Lambdakalkül müsste folgendes funktionieren:\\
\begin{align*}
	&    (\lambda x.xxx)(\lambda x.xxx)\\
	\Rightarrow&(\lambda x.xxx)(\lambda x.xxx)(\lambda x.xxx)\\
\end{align*}
%So sollte auch die Bedingung erfüllt sein, dass die Reduktion den Ausdruck größer macht.\\
%Ich weiß allerdings nicht genau, wie das in getyptem Lambdakalkül aussehen müsste (wobei das ja auch nicht explizit gefragt ist).. ;)
\section*{Aufgabe 4}
Schreiben Sie je einen getypten $\lambda$-Ausdruck für folgende Aufgaben:
\begin{compactenum}
\item Eine symmetrische Funktion soll dreifach auf ein Argument angewendet werden.
\begin{align*}
	\lambda fx.f(f(f(x)))&& [D \rightarrow D] \rightarrow D \rightarrow D
\end{align*}
\item Gegeben sei eine Liste der Länge $4$ von Elementen des Typs $\text{\lstinline!D!}$ und eine Funktion vom Typ $[\text{\lstinline!D!} \rightarrow \text{\lstinline!D!}]$, berechne die Anwendung dieser Funktion auf alle Listenelemente.
\begin{align*}
	\lambda L\ g.<g(\pi_1L);g(\pi_2L);g(\pi_3L);g(\pi_4L)>&& [D] \rightarrow [D\rightarrow D] \rightarrow [D]
\end{align*}
\item Beschreibe den unschönfinkel-Operator im getypten $\lambda$-Kalkül, der angewendet auf eine Funktion vom Typ $[\text{\lstinline!D!}_1 \rightarrow [\text{\lstinline!D!}_2 \rightarrow \text{\lstinline!D!}_3]]$ eine Funktion des Typs $[(\text{\lstinline!D!}_1 \times \text{\lstinline!D!}_2) \rightarrow \text{\lstinline!D!}_3]$ liefert, wobei für alle \lstinline!f!, \lstinline!a! und \lstinline!b!
\begin{lstlisting}
(unschönfinkel f) <a,b> = f a b
\end{lstlisting}
gelten soll.
\begin{align*}
	\lambda f. \lambda T.(f(\pi_1\ T) (\pi_2\ T))&& [D_1 \rightarrow [D_2 \rightarrow D_3] \rightarrow [(D_1\times D_2)\rightarrow D_3]
\end{align*}
\end{compactenum}

\end{document}
