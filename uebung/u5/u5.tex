\documentclass[ngerman,a4paper]{report}
\usepackage[ngerman]{babel}
\usepackage[T1]{fontenc}
\usepackage[utf8]{inputenc}
\usepackage{MyriadPro}
\usepackage[scaled]{beramono}
\newcommand\Small{\fontsize{10.5}{10.5}\selectfont}
\newcommand*\LSTfont{\Small\ttfamily\SetTracking{encoding=*}{-20}\lsstyle}
\usepackage{microtype}
\usepackage{geometry}
\geometry{verbose,tmargin=3cm,bmargin=3cm,lmargin=3cm,rmargin=3cm}
\usepackage{centernot}
\usepackage{listings}
%\usepackage{ stmaryrd }
\usepackage{mathtools}
\usepackage{paralist}
\usepackage{array}
\usepackage{color}
\usepackage{graphicx}
\usepackage{caption}
\usepackage{url}
\usepackage{amsmath}
\usepackage{accents}
\usepackage{tikz}

% Code Listing Style
\definecolor{darkblue}{rgb}{0,0,.6}
\definecolor{darkgreen}{rgb}{0,0.5,0}
\definecolor{darkred}{rgb}{0.5,0,0}
\lstset{%
	language=C,
	basicstyle=\LSTfont,
	commentstyle=\color{darkgreen},
	keywordstyle=\color{darkblue}\bfseries,
	breaklines=true,
	tabsize=2,
	xleftmargin=\fboxsep,
	xrightmargin=-\fboxsep,
	numbers=left,
	numberstyle=\tiny\color{gray},
	stepnumber=1,
	numbersep=5pt,
	frame=bt,
	stringstyle=\color{darkred},
	showstringspaces=false,
	rulecolor= \color{gray},
	emph=[1]%
	{%
	    then, not%
	},
	emphstyle=[1]{\color{darkblue}\bfseries},
	emph=[2]%
	{%  Datatypes
	    %
	},
	emphstyle=[2]{\color{darkblue}\bfseries},
	emph=[3]%
	{%
	    %
	},
	emphstyle=[3]{\color{darkred}\bfseries},
	literate=%
	{Ö}{{\"O}}1
	{Ä}{{\"A}}1
	{Ü}{{\"U}}1
	{ß}{{\ss}}2
	{ü}{{\"u}}1
	{ä}{{\"a}}1
	{ö}{{\"o}}1
}
\providecommand{\tabularnewline}{\\}

\usepackage{fancyhdr}
\pagestyle{fancy}
\usepackage{lastpage}
\makeatletter

\lhead{\textbf{\@title Tutor:} Paul Podlech \\ \@author}
\chead{}
\rhead{\@date \\ \thepage \ von \pageref{LastPage} }
\cfoot{}
%\cfoot{\small \textbf{Disclaimer:} Einige Lösungen wurden mit einer anderen Übungsgruppe (Jens Fischer, Johannes Dillmann, Tobias Famulla) inhaltlich diskutiert,  eine gewisse Ähnlichkeit der Lösungen ist möglich. Trotzdem sind alle Lösungen selbstständig von den hier genannten Mitgliedern erarbeitet.}
\renewcommand{\labelenumi}{\alph{enumi})}
\renewcommand{\maketitle}{}
\newcommand{\utilde}[1]{\underaccent{\tilde}{#1}}
\renewcommand{\familydefault}{\sfdefault}

\author{Florian Ritzel, Hinnerk van Bruinehsen, Tobias Höppner}
\title{SvP - Übung 05. }
\date{Sommersemester 2014}
\begin{document}
\maketitle
\section*{Aufgabe 1}
Ändern Sie die Sprache \textbf{WHILE} ab, indem Sie anstelle des atomaren Ausdruckes \lstinline!read! Anweisungen der Form \lstinline!read I! zulassen. Die Semantik dieser Anweisung lautet informell: Die Ausführung von \lstinline!read I! bewirkt eine Zuweisung des nächsten Eingabewertes an die Variable \lstinline!I! und eine Verkürzung der Eingabedatei um das erste Element.\\
Formalisieren Sie die Semantik von \lstinline!read I! denotationell.\\

\textbf{Idee von Flo:}\\

C[[read I]] (s,e,a) = { (S[n/I],e',a) mit e = n.e'\\

Soll heißen: der nächste gelesene Eingabewert (n) ersetzt (I) im Speicher. (e') ist ein um den ersten

Wert kürzeres (e). Naja....
\section*{Aufgabe 2}
Erweitern Sie die Sprache \textbf{WHILE} um Anweisungen der Form 
\begin{lstlisting}
for I := T_1 to T_2 do C
\end{lstlisting}.
Formalisieren Sie die Semantik dieser Anweisungen denotationell.\\

\textbf{Idee von Flo:}\\

Das soll doch am Ende ne ganz simple Zählvariable werden denk ich mal, also:\\


C[[while I:/=T2 do C]] (s,e,a) = { C[[C; while I:/=T2 do C; I:= I+1]](s,e',a)\\

Solange I noch nicht T2 entspricht, wird das Programm weiter ausgeführt.\\
Erweitern Sie die Sprache WHILE um den atomaren booleschen Term \lstinline!eof!. Die informelle Semantik von eof lautet: eof ist wahr gdw die Eingabe leer ist.\\
Formalisieren Sie die Semantik von \lstinline!eof! denotationell.\\

\textbf{Idee von Flo:}\\

B[[eof]] s,e = { (falsch, e'), falls e = b.e' mit b e BOOL oder e = E oder e=n.e' mit n e Zahl\\
B[[eof]] s,e = { (wahr, e'), sonst\\

Ich hoffe ich habe alle möglichen Eingabearten abgegriffen.
\section*{Aufgabe 4}
Programmieren Sie in WHILE (einschließlich \lstinline!eof!) einen Algorithmus zur Berechnung der Summe aller Eingabewerte. Beweisen Sie die Korrektheit Ihres Programms anhand der denotationellen Semantik. Diskutieren Sie die Problematik beim Fehlen von \lstinline!eof!.
\end{document}
