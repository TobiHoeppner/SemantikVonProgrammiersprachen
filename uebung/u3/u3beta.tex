\documentclass[ngerman,a4paper]{report}
\usepackage[ngerman]{babel}
\usepackage[T1]{fontenc}
\usepackage[utf8]{inputenc}
%\usepackage{MyriadPro}
%\usepackage[scaled]{beramono}
\newcommand\Small{\fontsize{10.5}{10.5}\selectfont}
\newcommand*\LSTfont{\Small\ttfamily\SetTracking{encoding=*}{-20}\lsstyle}
\usepackage{microtype}
\usepackage{geometry}
\geometry{verbose,tmargin=3cm,bmargin=3cm,lmargin=3cm,rmargin=3cm}
\usepackage{listings}
\usepackage{stmaryrd}
\usepackage{paralist}
\usepackage{array}
\usepackage{color}
\usepackage{graphicx}
\usepackage{caption}
\usepackage{url}
\usepackage{amsmath}
\usepackage{accents}
\usepackage{tikz}
\usepackage[T1]{fontenc}
\usepackage{lmodern}

% Code Listing Style
\definecolor{darkblue}{rgb}{0,0,.6}
\definecolor{darkgreen}{rgb}{0,0.5,0}
\definecolor{darkred}{rgb}{0.5,0,0}
\lstset{%
	language=C, 
	basicstyle=\LSTfont,
	commentstyle=\color{darkgreen}, 
	keywordstyle=\color{darkblue}\bfseries, 
	breaklines=true,
	tabsize=2,
	xleftmargin=\fboxsep,
	xrightmargin=-\fboxsep,
	numbers=left,
	numberstyle=\tiny\color{gray},
	stepnumber=1,
	numbersep=5pt,
	frame=bt,
	stringstyle=\color{darkred},
	showstringspaces=false,
	rulecolor= \color{gray},
	emph=[1]%
	{%  
	    then, not%
	},
	emphstyle=[1]{\color{darkblue}\bfseries},
	emph=[2]%
	{%  Datatypes
	    %
	},
	emphstyle=[2]{\color{darkblue}\bfseries},
	emph=[3]%
	{%  
	    %
	},
	emphstyle=[3]{\color{darkred}\bfseries},
	literate=%
	{Ö}{{\"O}}1
	{Ä}{{\"A}}1
	{Ü}{{\"U}}1
	{ß}{{\ss}}2
	{ü}{{\"u}}1
	{ä}{{\"a}}1
	{ö}{{\"o}}1
}
\providecommand{\tabularnewline}{\\}

\usepackage{fancyhdr}
\pagestyle{fancy}
\usepackage{lastpage}
\makeatletter

\lhead{\textbf{\@title Tutor:} Paul Podlech \\ \@author}
\chead{}
\rhead{\@date \\ \thepage \ von \pageref{LastPage} }

\renewcommand{\maketitle}{}
\newcommand{\utilde}[1]{\underaccent{\tilde}{#1}}
\renewcommand{\familydefault}{\sfdefault}
 
\author{Florian Ritzel, Hinnerk van Bruinehsen, Tobias Höppner}
\title{SvP - Übung 03. }
\date{Sommersemester 2014}

\begin{document} 
\maketitle 
\section*{Aufgabe 1}
Die Syntax von WHILE sei um die Regel\\
$C::= repeat\ C\ until\ B$\\
erweitert. Ergänzen Sie die operationelle Semantik von WHILE, so dass diese zusätzliche Anweisungsform angemessen behandelt wird.\\\\\\
$\Delta\langle W|S|repeat\ C\ until\ B.K|E|A\rangle = \langle W|S|C.while\ (not\ B)\ do\ C.K|E|A\rangle$\\\\
C soll mindestens einmal ausgeführt werden. Danach verhält sich die repeat-Schleife genau wie die while-Schleife.

\section*{Aufgabe 2}

Erweitern Sie die Syntax von \emph{WHILE}, so dass in den boolschen Ausdrücken auch die boolschen Operatoren \textit{and} und \textit{or} vorkommen dürfen. Geben Sie für diese Erweiterung eine operationelle Semantik an, die eine nicht-strikte Semantik von \textit{and} und \textit{or} festlegt.\\\\\\
$B::=W|not\ B|\ T_1\ BOP\ T_2| B_1\ and\  B_2| B_1\ or\ B_2| read$\\\\
\textbf{and:}\\\\
$\Delta\langle W|S|B_1\ and\ B_2.K|E|A\rangle = \langle W|S|B_1 and.B_2.K|E|A\rangle$\\
$\Delta\langle true.W|S|and.B_2.K|E|A\rangle = \langle W|S|B_2.K|E|A\rangle$\\
$\Delta\langle false.W|S|and.B_2.K|E|A\rangle = \langle false.W|S|K|E|A\rangle$\\\\
\textbf{or:}\\\\
$\Delta\langle W|S|B_1\ or\ B_2.K|E|A\rangle = \langle W|S|B_1.or.B_2.K|E|A\rangle$\\
$\Delta\langle false.W|S|or.B_2.K|E|A\rangle = \langle W|S|B_2.K|E|A\rangle$\\
$\Delta\langle true.W|S|or.B_2.K|E|A\rangle = \langle true.W|S|K|E|A\rangle$\\

\section*{Aufgabe 3}
Formulieren Sie informell eine Präzisierung der angegebenen \emph{WHILE}-Semantik, die die genannten Fehlerquellen behandelt.\\\\\\
\textbf{Bereichsüberschreitung:}\\
Der gültige Wertebereich wird verlassen, Min/Max werden also unter- bzw. überschritten. Ergebnis ist ein gewöhnlicher Overflow, da der Nachfolger von Max Min ist und der Vorgänger von Min Max.\\\\
\textbf{Division durch Null:}\\
Division durch Null ist nicht definiert, also wird ein Fehler geworfen und das Programm abgebrochen.\\\\
\textbf{Berechnung von read bei leerer Eingabedatei:}\\
Im Falle einer leeren Eingabe liest read für B false und für Z 0.\\\\
\textbf{Typkonflikte} (drei Möglichkeiten):\\\\
\textbf{1.}
Kompatibilitätsprüfung von T1 und T2 und Überprüfung ob die Operation OP für T1 und T2 definiert ist. Ist eines davon nicht der Fall wird das Programm abgebrochen.\\\\
\textbf{2.}
Wird ein B erwartet, aber ein T oder ein Z gelesen, wird = als \textit{false} und alles andere als \textit{true} verstanden.\\\\
\textbf{3.} Jeder Typfehler führt zu einem Fehler und das PRogramm wird an der Stelle abgebrochen.

\section*{Aufgabe 4}
Implementieren Sie in einer Sprache Ihrer Wahl\\

- den Zustandsraum der WSKEA-Maschine,\\

- eine Funktion anfang, die zu einem WHILE-Programm und einer Eingabe

den Anfangszustand ergibt, und\\

- die Zustandsüberfüuhrungsfunktion delta.\\\\
!!under construction!!
\end{document}
