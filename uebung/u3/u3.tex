\documentclass[ngerman,a4paper]{report}
\usepackage[ngerman]{babel}
\usepackage[T1]{fontenc}
\usepackage[utf8]{inputenc}
\usepackage{MyriadPro}
\usepackage[scaled]{beramono}
\newcommand\Small{\fontsize{10.5}{10.5}\selectfont}
\newcommand*\LSTfont{\Small\ttfamily\SetTracking{encoding=*}{-20}\lsstyle}
\usepackage{microtype}
\usepackage{geometry}
\geometry{verbose,tmargin=3cm,bmargin=3cm,lmargin=3cm,rmargin=3cm}
\usepackage{centernot}
\usepackage{listings}
%\usepackage{ stmaryrd }
\usepackage{mathtools}
\usepackage{paralist}
\usepackage{array}
\usepackage{color}
\usepackage{graphicx}
\usepackage{caption}
\usepackage{url}
\usepackage{amsmath}
\usepackage{accents}
\usepackage{tikz}

% Code Listing Style
\definecolor{darkblue}{rgb}{0,0,.6}
\definecolor{darkgreen}{rgb}{0,0.5,0}
\definecolor{darkred}{rgb}{0.5,0,0}
\lstset{%
	language=C,
	basicstyle=\LSTfont,
	commentstyle=\color{darkgreen},
	keywordstyle=\color{darkblue}\bfseries,
	breaklines=true,
	tabsize=2,
	xleftmargin=\fboxsep,
	xrightmargin=-\fboxsep,
	numbers=left,
	numberstyle=\tiny\color{gray},
	stepnumber=1,
	numbersep=5pt,
	frame=bt,
	stringstyle=\color{darkred},
	showstringspaces=false,
	rulecolor= \color{gray},
	emph=[1]%
	{%
	    then, not%
	},
	emphstyle=[1]{\color{darkblue}\bfseries},
	emph=[2]%
	{%  Datatypes
	    %
	},
	emphstyle=[2]{\color{darkblue}\bfseries},
	emph=[3]%
	{%
	    %
	},
	emphstyle=[3]{\color{darkred}\bfseries},
	literate=%
	{Ö}{{\"O}}1
	{Ä}{{\"A}}1
	{Ü}{{\"U}}1
	{ß}{{\ss}}2
	{ü}{{\"u}}1
	{ä}{{\"a}}1
	{ö}{{\"o}}1
}
\providecommand{\tabularnewline}{\\}

\usepackage{fancyhdr}
\pagestyle{fancy}
\usepackage{lastpage}
\makeatletter

\lhead{\textbf{\@title Tutor:} Paul Podlech \\ \@author}
\chead{}
\rhead{\@date \\ \thepage \ von \pageref{LastPage} }
\cfoot{}
%\cfoot{\small \textbf{Disclaimer:} Einige Lösungen wurden mit einer anderen Übungsgruppe (Jens Fischer, Johannes Dillmann, Tobias Famulla) inhaltlich diskutiert,  eine gewisse Ähnlichkeit der Lösungen ist möglich. Trotzdem sind alle Lösungen selbstständig von den hier genannten Mitgliedern erarbeitet.}

\renewcommand{\maketitle}{}
\newcommand{\utilde}[1]{\underaccent{\tilde}{#1}}
\renewcommand{\familydefault}{\sfdefault}

\author{Florian Ritzel, Hinnerk van Bruinehsen, Tobias Höppner}
\title{SvP - Übung 03. }
\date{Sommersemester 2014}

\begin{document}
\maketitle
\section*{Aufgabe 1}
Die Syntax von WHILE sei um die Regel
\begin{lstlisting}
C::= repeat C until B
\end{lstlisting}
erweitert. Ergänzen Sie die operationelle Semantik von WHILE, so dass diese zusätzliche Anweisungsform angemessen behandelt wird.\\

$\Delta\langle W|S|\underline{repeat}\ C\ \underline{until}\ B.K|E|A\rangle = \langle W|S|C.\underline{while}\ (\underline{not} B)\ \underline{do}\ C.K|E|A\rangle$\\
C wird mindestens einmal ausgeführt, anschliessend verhält sich die \glqq repeat C until B\grqq -Schleife identisch zu \glqq while $\neg$B do C \grqq .\\

\section*{Aufgabe 2}
Erweitern Sie die Syntax von WHILE, so dass in den boolschen Ausdrücken auch die boolschen Operatoren \textit{and} und \textit{or} vorkommen dürfen. Geben Sie für diese Erweiterung eine operationelle Semantik an, die eine nicht-strikte Semantik von \textit{and} und \textit{or} festlegt.\\

$B::=W| \underline{not}\ B| T_1\ BOP\ T_2| B_1\ \underline{and}\  B_2| B_1\ \underline{or}\ B_2| \underline{read}$\\

\textbf{And:}\\
$\Delta\langle W|S|B_1\ \underline{and}\ B_2.K|E|A\rangle = \langle W|S|B_1.\underline{and}.B_2.K|E|A\rangle$\\
$\Delta\langle \underline{true}.W|S|\underline{and}.B_2.K|E|A\rangle = \langle W|S|B_2.K|E|A\rangle$\\
$\Delta\langle \underline{false}.W|S|\underline{and}.B_2.K|E|A\rangle = \langle \underline{false}.W|S|K|E|A\rangle$\\

\textbf{Or:}\\
$\Delta\langle W|S|B_1\ \underline{or}\ B_2.K|E|A\rangle = \langle W|S|B_1.\underline{or}.B_2.K|E|A\rangle$\\
$\Delta\langle \underline{false}.W|S|\underline{or}.B_2.K|E|A\rangle = \langle W|S|B_2.K|E|A\rangle$\\
$\Delta\langle \underline{true}.W|S|\underline{or}.B_2.K|E|A\rangle = \langle \underline{true}.W|S|K|E|A\rangle$\\

\section*{Aufgabe 3}
Erweitern Sie die WSKEA-Maschine um eine Komponente $N$ für Nachrichten (Texte), in der kurze, sinnvolle Meldungen eingetragen werden, wenn es keinen Folgezustand gibt, oder wenn die Ausführung korrekt terminiert.\\

Grundsätzlich identisch zur WSKEA-Maschine\\

Grundzustand der WSKEAN-Maschine ist: $\langle W|S|K|E|A|\epsilon\rangle$\\

Falls das Programm korrekt terminiert gilt: $\Delta\langle W|S|\epsilon|E|A|\epsilon\rangle = \langle W|S|\epsilon|E|A|\text{\glqq Terminiert\grqq} \rangle$\\

In einem Fehlerzustand gibt die Maschine eine informative Fehlermeldung zurück. Ein Beispiel wäre:\\

$\Delta\langle 0.n.W|S|/.K|E|A|\epsilon\rangle = \langle W|S|K|E|A|\text{\glqq Fehler: Division durch 0\grqq}\rangle$\\

Die Darstellung weiterer Fehlerzustände ist analog dazu.

\section*{Aufgabe 4 \text{(freiwillig)}}
Implementieren Sie in einer Sprache Ihrer Wahl
\begin{itemize}
	\item den Zustandsraum der WSKEA-Maschine,
	\item eine Funktion \texttt{anfang}, die zu einem WHILE-Programm und einer Eingabe den Anfangszustand ergibt, und
	\item die Zustandsüberführungsfunktion \texttt{delta}.
\end{itemize}

Lösung in \textbf{Whitespace} ;-) :
\begin{lstlisting}[escapechar=\%]






















































%
\end{lstlisting}
\newpage
Lösungsidee in \textbf{Python}:
\lstinputlisting[language=Python]{py/wskea/main.py}
\newpage
Ausgabe:
\begin{lstlisting}
run 0: [['x', 42], [], ['output', '=', 1, 1, 'not', True, 'add', 'read', 'read', 'skip', 'assign'], [1, 1], []]
run 1: [[], [(42, 'x')], ['output', '=', 1, 1, 'not', True, 'add', 'read', 'read', 'skip'], [1, 1], []]
run 2: [[], [(42, 'x')], ['output', '=', 1, 1, 'not', True, 'add', 'read', 'read'], [1, 1], []]
run 3: [[1], [(42, 'x')], ['output', '=', 1, 1, 'not', True, 'add', 'read'], [1], []]
run 4: [[1, 1], [(42, 'x')], ['output', '=', 1, 1, 'not', True, 'add'], [], []]
run 5: [[2], [(42, 'x')], ['output', '=', 1, 1, 'not', True], [], []]
run 6: [[2, True], [(42, 'x')], ['output', '=', 1, 1, 'not'], [], []]
run 7: [[2, False], [(42, 'x')], ['output', '=', 1, 1], [], []]
run 8: [[2, False, 1], [(42, 'x')], ['output', '=', 1], [], []]
run 9: [[2, False, 1, 1], [(42, 'x')], ['output', '='], [], []]
run 10: [[2, False, True], [(42, 'x')], ['output'], [], []]
run 11: [[2, False], [(42, 'x')], [], [], [True]]
success after 12 iterations !
\end{lstlisting}
Kommentar zur Lösung:\\
Die Lösung nutzt einige Annehmlichkeiten von Python: Listen und Typen. Trotzdem sind nicht alle Konstrukte implementiert. Bisher habe ich noch keine Idee, wie ich $I, C, T, \underline{BOP}, \underline{OP}$ in Python repräsentiere, daher sind einige Ausdrücke in diesem Programm nicht ausführbar. Damit wird es sehr schwierig Strukturen wie \lstinline!if B then C1 else C2! oder \lstinline!while B do C! zu implementieren und damit die Aufgabe nicht vollständig bearbeitet.
\end{document}
