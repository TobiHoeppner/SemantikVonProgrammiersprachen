\documentclass[ngerman,a4paper]{report}
\usepackage[ngerman]{babel}
\usepackage[T1]{fontenc}
\usepackage[utf8]{inputenc}
\usepackage{lmodern}
\usepackage{MyriadPro}
\usepackage[scaled]{beramono}
\newcommand\Small{\fontsize{10.5}{10.5}\selectfont}
\newcommand*\LSTfont{\Small\ttfamily\SetTracking{encoding=*}{-20}\lsstyle}
\usepackage{microtype}
\usepackage{geometry}
\geometry{verbose,tmargin=3cm,bmargin=3cm,lmargin=3cm,rmargin=3cm}
\usepackage{centernot}
\usepackage{listings}
\usepackage{paralist}
%\usepackage{array}
\usepackage{xcolor}
%\usepackage{graphicx}
%\usepackage{caption}
%\usepackage{url}
%\usepackage[verification]{struktex}
\usepackage{amsmath}
\usepackage{amsfonts}
\usepackage{mathtools}
%\usepackage{accents}
\usepackage{tikz}
\usetikzlibrary{shapes,arrows,automata}
\tikzset{
	treenode/.style = {align=center, inner sep=1pt, text centered,font=\sffamily},
	node/.style = {treenode, font=\sffamily\bfseries, text width=1.5em},
	cloud/.style = {draw, circle, fill=red!20, node distance=3cm, minimum height=2em},
	decision/.style = {diamond, draw, fill=blue!20, text width=6em, text badly centered, node distance=3cm, inner sep=0pt},
	block/.style = {rectangle, draw, fill=green!20, text width=6em, text centered, rounded corners, minimum height=4em},
	line/.style = {draw, -latex'}
}
\usepackage{stmaryrd}

% Code Listing Style
\definecolor{darkblue}{rgb}{0,0,.6}
\definecolor{darkgreen}{rgb}{0,0.5,0}
\definecolor{darkred}{rgb}{0.5,0,0}
\lstset{%
	language=C,
	basicstyle=\LSTfont,
	commentstyle=\color{darkgreen},
	keywordstyle=\color{darkblue}\bfseries,
	breaklines=true,
	tabsize=2,
	xleftmargin=\fboxsep,
	xrightmargin=-\fboxsep,
	numbers=left,
	numberstyle=\tiny\color{gray},
	stepnumber=1,
	numbersep=5pt,
	frame=bt,
	stringstyle=\color{darkred},
	showstringspaces=false,
	rulecolor= \color{gray},
	emph=[1]%
	{%
	    then, not%
	},
	emphstyle=[1]{\color{darkblue}\bfseries},
	emph=[2]%
	{%  Datatypes
	    %
	},
	emphstyle=[2]{\color{darkblue}\bfseries},
	emph=[3]%
	{%
	    %
	},
	emphstyle=[3]{\color{darkred}\bfseries},
	literate=%
	{Ö}{{\"O}}1
	{Ä}{{\"A}}1
	{Ü}{{\"U}}1
	{ß}{{\ss}}2
	{ü}{{\"u}}1
	{ä}{{\"a}}1
	{ö}{{\"o}}1
}
\providecommand{\tabularnewline}{\\}

\usepackage{fancyhdr}
\pagestyle{fancy}
\usepackage{lastpage}
\makeatletter

\lhead{\textbf{\@title Tutor:} Paul Podlech \\ \@author}
\chead{}
\rhead{\@date \\ \thepage \ von \pageref{LastPage} }
\cfoot{}
%\cfoot{\small \textbf{Disclaimer:} Einige Lösungen wurden mit einer anderen Übungsgruppe (Jens Fischer, Johannes Dillmann, Tobias Famulla) inhaltlich diskutiert,  eine gewisse Ähnlichkeit der Lösungen ist möglich. Trotzdem sind alle Lösungen selbstständig von den hier genannten Mitgliedern erarbeitet.}
\renewcommand{\labelenumi}{\alph{enumi})}
\renewcommand{\maketitle}{}
\newcommand{\utilde}[1]{\underaccent{\tilde}{#1}}
\renewcommand{\familydefault}{\sfdefault}

\author{Florian Ritzel, Hinnerk van Bruinehsen, Tobias Höppner}
\title{SvP - Übung 07. }
\date{Sommersemester 2014}
\begin{document}
\maketitle
\section*{Aufgabe 1}
\begin{compactenum}
\item[a)] Geben Sieein Beispiel für eine nicht stetige Funktion $f$ über \lstinline{cpo}'s an.\\
\item[b)] Beweisen Sie, dass die Komposition stetiger Funktionen wieder eine stetige Funktion ergibt.\\
\end{compactenum}
\section*{Aufgabe 2}
\begin{compactenum}
\item[a)] Zeigen Sie, wie Sie zu gegebenen \lstinline{cpo}s $D_1,...,D_n$ mit $n\geq 2$ den Bereich der disjunkten Vereinigung $(D_1 + ... + D_n)$ erklären können, ohne die minimalen Elemente zu verschmelzen.\\
\item[b)] Definieren Sie folgende Injektions-, Projektions- und Testfunktionen in kanonischer Weise:\\
	\begin{align*}
		in_i: D_i \rightarrow (D_1+...+D_n)& \text{ für alle } 1\leq i\leq n\\
		out_i: (D_1+...+D_n)\rightarrow D_i& \text{ für alle } 1\leq i\leq n\\
		is_i: (D_1+...+D_n)\rightarrow BOOL_\bot& \text{ für alle } 1\leq i\leq n\\
	\end{align*}
\end{compactenum}
\section*{Aufgabe 3}
Definieren Sie stetige Erweiterungen der Addition und des Tests auf Gleichheit, so dass diese Operationen total werden auf den \lstinline{cpo}'s $\mathbb{N}_\bot$ und $BOOL_\bot$. Diskutieren Sie, ob es mehrere solche Erweiterungen gibt.
\section*{Aufgabe 4}
Seien $D_1$ und $D_2$ \lstinline{cpo}'s und auf $f:D_1\rightarrow D_2$ und $d:D_2 \rightarrow D_1$ stetige Funktionen.\\
Beweisen Sie:\\
\begin{align*}
	fix_{f\circ g} = f(fix_{g \circ f})& \text{ und }\\
	fix_{g\circ f} = g(fix_{f\circ g}&\\
\end{align*}
\end{document}
