\documentclass[ngerman,a4paper]{report}
\usepackage[ngerman]{babel}
\usepackage[T1]{fontenc}
\usepackage[utf8]{inputenc}
\usepackage{lmodern}
\usepackage{MyriadPro}
\usepackage[scaled]{beramono}
\newcommand\Small{\fontsize{10.5}{10.5}\selectfont}
\newcommand*\LSTfont{\Small\ttfamily\SetTracking{encoding=*}{-20}\lsstyle}
\usepackage{microtype}
\usepackage{geometry}
\geometry{verbose,tmargin=3cm,bmargin=3cm,lmargin=3cm,rmargin=3cm}
\usepackage{centernot}
\usepackage{listings}
\usepackage{paralist}
%\usepackage{array}
\usepackage{xcolor}
%\usepackage{graphicx}
%\usepackage{caption}
%\usepackage{url}
%\usepackage[verification]{struktex}
\usepackage{amsmath}
\usepackage{amsfonts}
\usepackage{mathtools}
%\usepackage{accents}
\usepackage{tikz}
\usetikzlibrary{shapes,arrows,automata}
\tikzset{
	treenode/.style = {align=center, inner sep=1pt, text centered,font=\sffamily},
	node/.style = {treenode, font=\sffamily\bfseries, text width=1.5em},
	cloud/.style = {draw, circle, fill=red!20, node distance=3cm, minimum height=2em},
	decision/.style = {diamond, draw, fill=blue!20, text width=6em, text badly centered, node distance=3cm, inner sep=0pt},
	block/.style = {rectangle, draw, fill=green!20, text width=6em, text centered, rounded corners, minimum height=4em},
	line/.style = {draw, -latex'}
}
\usepackage{stmaryrd}

% Code Listing Style
\definecolor{darkblue}{rgb}{0,0,.6}
\definecolor{darkgreen}{rgb}{0,0.5,0}
\definecolor{darkred}{rgb}{0.5,0,0}
\lstset{%
	language=C,
	basicstyle=\LSTfont,
	commentstyle=\color{darkgreen},
	keywordstyle=\color{darkblue}\bfseries,
	breaklines=true,
	tabsize=2,
	xleftmargin=\fboxsep,
	xrightmargin=-\fboxsep,
	numbers=left,
	numberstyle=\tiny\color{gray},
	stepnumber=1,
	numbersep=5pt,
	frame=bt,
	stringstyle=\color{darkred},
	showstringspaces=false,
	rulecolor= \color{gray},
	emph=[1]%
	{%
	    then, not%
	},
	emphstyle=[1]{\color{darkblue}\bfseries},
	emph=[2]%
	{%  Datatypes
	    %
	},
	emphstyle=[2]{\color{darkblue}\bfseries},
	emph=[3]%
	{%
	    %
	},
	emphstyle=[3]{\color{darkred}\bfseries},
	literate=%
	{Ö}{{\"O}}1
	{Ä}{{\"A}}1
	{Ü}{{\"U}}1
	{ß}{{\ss}}2
	{ü}{{\"u}}1
	{ä}{{\"a}}1
	{ö}{{\"o}}1
}
\providecommand{\tabularnewline}{\\}

\usepackage{fancyhdr}
\pagestyle{fancy}
\usepackage{lastpage}
\makeatletter

\lhead{\textbf{\@title Tutor:} Paul Podlech \\ \@author}
\chead{}
\rhead{\@date \\ \thepage \ von \pageref{LastPage} }
\cfoot{}
%\cfoot{\small \textbf{Disclaimer:} Einige Lösungen wurden mit einer anderen Übungsgruppe (Jens Fischer, Johannes Dillmann, Tobias Famulla) inhaltlich diskutiert,  eine gewisse Ähnlichkeit der Lösungen ist möglich. Trotzdem sind alle Lösungen selbstständig von den hier genannten Mitgliedern erarbeitet.}
\renewcommand{\labelenumi}{\alph{enumi})}
\renewcommand{\maketitle}{}
\newcommand{\utilde}[1]{\underaccent{\tilde}{#1}}
\renewcommand{\familydefault}{\sfdefault}

\author{Florian Ritzel, Hinnerk van Bruinehsen, Tobias Höppner}
\title{Anonyme Semantiker - Übung 07. }
\date{Sommersemester 2014}
\begin{document}
\maketitle
\section*{Aufgabe 1}
\begin{compactenum}
\item[a)] Geben Sie ein Beispiel für eine nicht stetige Funktion $f$ über \lstinline!cpo!'s an.\\
%\textbf{Idee von Tobi:}\\
Definition der Stetigkeit aus VL: Seinen $A$ und $B$ \lstinline!cpo!'s\\
Eine Funktion $f: A \rightarrow B$ heißt \textbf{stetig}, wenn $f(K)$ eine Kette in $B$ ist und $f(\bigsqcup K) = \bigsqcup f(K)$ für alle $K \subseteq A$ mit $K$ ist Kette in $A$.\\

Noch ein Blick auf die Kette aus der Vorlesung:\\
$K$ ist Kette, wenn zu je zwei $k_1, k_2 \in K$ gilt: $k_1 \sqsubseteq_A k_2$ oder $k_2 \sqsubseteq_A k_1$.\\
%Also der Vorgänger steht mit dem Nachfolger oder der Nachfolger steht mit dem Vorgänger irgendwie in Relation.\\

\lstinline!cpo!: $(\mathbb{N}_\infty,\leq)$\\
$f(x) = x \mod 2$ mit $x \in \mathbb{N}$\\
Ist eine nicht stetige Funktion.\\
%Also sowas wie $f(x) = x \mod 2$ mit $x \in \mathbb{N}$ sollte dem Widersprechen, oder?\\
%
%\textbf{Hinnerk:}
%Was ist mit sowas:\\
%\lstinline!cpo!: $(\mathbb{N}_\infty,\leq)$\\
%\begin{align*}
%f(x) = \begin{cases}
%	1, & \text{wenn } x = 42 \\
%    x, & sonst
%\end{cases}
%\end{align*}


\item[b)] Beweisen Sie, dass die Komposition stetiger Funktionen wieder eine stetige Funktion ergibt.\\
%\textbf{Idee von Tobi:} na dit is relativ simple im Kopf, aber unklar wie ich es aufschreibe..\\
Skizze: Seien $f : A \rightarrow B$ und $g : B \rightarrow C$ stetige Funktionen und $A, B$ und $C$ \lstinline!cpo!'s und der $\circ$-Operator steht - wie üblich - für die Komposition.\\
%\begin{align*}
%f &: A \rightarrow B &\\
%g &: B \rightarrow C &\\
%\text{ also ist } &\\
%g \circ f &: (A \rightarrow B) \rightarrow C = A \rightarrow C &\\
%\text{ ebenfalls stetig!}\\
%\end{align*}
Für alle Ketten $k \subseteq A$:
\begin{align*}
(g \circ f) (\bigsqcup k) &= \bigsqcup (g \circ f) (k) \\
&= \bigsqcup g (f (k)) \\
\end{align*}
Wegen Monotonie von $f$ ist $f(k) \subseteq B$ eine Kette und wegen der Monotonie von $g$ ist $g(f(k)) \subseteq B$ Kette in C.
\begin{align*}
\bigsqcup g (f()k) &= g (\bigsqcup f(k))\\
&= g (f (\bigsqcup k))\\
&= (g \circ f) (\bigsqcup K)\\
&& q.e.d.\\
\end{align*}
\end{compactenum}

\section*{Aufgabe 2}
\begin{compactenum}
\item[a)] Zeigen Sie, wie Sie zu gegebenen \lstinline!cpo!s $D_1,...,D_n$ mit $n\geq 2$ den Bereich der disjunkten Vereinigung $(D_1 + ... + D_n)$ erklären können, ohne die minimalen Elemente zu verschmelzen.\\

	$D:= D_1+...+S_2 = (\lbrace (d,i)| 1\leq i \leq nm d \in D_i \rbrace \cup \bot_0, \sqsubseteq_0)$\\
	$\bot_0 \sqsubseteq d \in D$\\
	$(d,i) \sqsubseteq_D (d',j)$ genau dann wenn $ i=j$, und $ d\sqsubseteq_{0_i}d'$
%\textbf{Idee:}
%Bei der Vereinigung disjunkter cpos wird der entstehende Wertebereich so groß, wie der gemeinsame Wertebereich aller cpos mit zusätzlich einem weiteren Element: ein gemeinsames $\bot$ (es kann ja kein Element $\bot$ sein, dass vorher schon in der Menge enthalten war, da die cpos disjunkt waren).
Da jedes $D_i$ vorher cpo war und nun jeweils ein weiteres, gemeinsames Element, dass jeweils die Kritierien für $\bot$ erfüllt, ist das Resultat auch ein cpo.
\item[b)] Definieren Sie folgende Injektions-, Projektions- und Testfunktionen in kanonischer Weise:\\
%	\begin{align*}
%		&in_i: D_i \rightarrow (D_1+...+D_n)& \text{ für alle } 1\leq i\leq n\\
%		&out_i: (D_1+...+D_n)\rightarrow D_i& \text{ für alle } 1\leq i\leq n\\
%		&is_i: (D_1+...+D_n)\rightarrow BOOL_\bot& \text{ für alle } 1\leq i\leq n\\
%	\end{align*}
%
%
	\begin{align*}
		&in_i:& D_i \rightarrow (D_1+...+D_n)& \text{ für alle } 1\leq i\leq n\\
		&& d \mapsto (d,i)& \text{ für } 1 \leq i \leq n\\
		&out_i:& (D_1+...+D_n)\rightarrow D_i& \text{ für alle } 1\leq i\leq n\\
		&& d \mapsto \begin{cases} d & \text{ falls } x=(d,i)\\ \bot_{D_i} & \text{ sonst }\end{cases}\\
		&is_i:& (D_1+...+D_n)\rightarrow BOOL_\bot& \text{ für alle } 1\leq i\leq n\\
		&& d \mapsto \begin{cases} \underline{true},& \text{ falls } x=(d,i)\\ \underline{false},& \text{ falls } x = (d,j), i \neq j\\ \bot,& \text{ sonst } \end{cases}
	\end{align*}


\end{compactenum}
\section*{Aufgabe 3}
Definieren Sie stetige Erweiterungen der Addition und des Tests auf Gleichheit, so dass diese Operationen total werden auf den \lstinline!cpo!'s $\mathbb{N}_\bot$ und $BOOL_\bot$. Diskutieren Sie, ob es mehrere solche Erweiterungen gibt.\\
%\textbf{Idee von Tobi:} Irgendwie macht das nicht viel Sinn mit dem Bool und der Addition\\
%Es sollen $+, =$ erweitert werden, sodass $(\mathbb{N}_\perp, +)$, $(BOOL_\perp, +)$ und $(\mathbb{N}_\perp, =)$, $(BOOL_\perp, =)$ neben relexiv, transitiv und antisymetrisch auch noch total sind.\\ Soweit ich heraus bekommen habe ist "total werden" eine Umschreibung von Kette bilden.\\
\section*{Aufgabe 4}
Seien $D_1$ und $D_2$ \lstinline!cpo!'s und auf $f:D_1\rightarrow D_2$ und $g:D_2 \rightarrow D_1$ stetige Funktionen.\\
Beweisen Sie:\\
\begin{align*}
	fix_{f\circ g}\quad &= \quad f(fix_{g \circ f}) \quad\quad \text{ und }\\
	fix_{g\circ f}\quad &= \quad g(fix_{f\circ g}) &\\
\end{align*}
\end{document}
