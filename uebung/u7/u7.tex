\documentclass[ngerman,a4paper]{report}
\usepackage[ngerman]{babel}
\usepackage[T1]{fontenc}
\usepackage[utf8]{inputenc}
\usepackage{lmodern}
\usepackage{MyriadPro}
\usepackage[scaled]{beramono}
\newcommand\Small{\fontsize{10.5}{10.5}\selectfont}
\newcommand*\LSTfont{\Small\ttfamily\SetTracking{encoding=*}{-20}\lsstyle}
\usepackage{microtype}
\usepackage{geometry}
\geometry{verbose,tmargin=3cm,bmargin=3cm,lmargin=3cm,rmargin=3cm}
\usepackage{centernot}
\usepackage{listings}
\usepackage{paralist}
%\usepackage{array}
\usepackage{xcolor}
%\usepackage{graphicx}
%\usepackage{caption}
%\usepackage{url}
%\usepackage[verification]{struktex}
\usepackage{amsmath}
\usepackage{amsfonts}
\usepackage{mathtools}
%\usepackage{accents}
\usepackage{tikz}
\usetikzlibrary{shapes,arrows,automata}
\tikzset{
	treenode/.style = {align=center, inner sep=1pt, text centered,font=\sffamily},
	node/.style = {treenode, font=\sffamily\bfseries, text width=1.5em},
	cloud/.style = {draw, circle, fill=red!20, node distance=3cm, minimum height=2em},
	decision/.style = {diamond, draw, fill=blue!20, text width=6em, text badly centered, node distance=3cm, inner sep=0pt},
	block/.style = {rectangle, draw, fill=green!20, text width=6em, text centered, rounded corners, minimum height=4em},
	line/.style = {draw, -latex'}
}
\usepackage{stmaryrd}

% Code Listing Style
\definecolor{darkblue}{rgb}{0,0,.6}
\definecolor{darkgreen}{rgb}{0,0.5,0}
\definecolor{darkred}{rgb}{0.5,0,0}
\lstset{%
	language=C,
	basicstyle=\LSTfont,
	commentstyle=\color{darkgreen},
	keywordstyle=\color{darkblue}\bfseries,
	breaklines=true,
	tabsize=2,
	xleftmargin=\fboxsep,
	xrightmargin=-\fboxsep,
	numbers=left,
	numberstyle=\tiny\color{gray},
	stepnumber=1,
	numbersep=5pt,
	frame=bt,
	stringstyle=\color{darkred},
	showstringspaces=false,
	rulecolor= \color{gray},
	emph=[1]%
	{%
	    then, not%
	},
	emphstyle=[1]{\color{darkblue}\bfseries},
	emph=[2]%
	{%  Datatypes
	    %
	},
	emphstyle=[2]{\color{darkblue}\bfseries},
	emph=[3]%
	{%
	    %
	},
	emphstyle=[3]{\color{darkred}\bfseries},
	literate=%
	{Ö}{{\"O}}1
	{Ä}{{\"A}}1
	{Ü}{{\"U}}1
	{ß}{{\ss}}2
	{ü}{{\"u}}1
	{ä}{{\"a}}1
	{ö}{{\"o}}1
}
\providecommand{\tabularnewline}{\\}

\usepackage{fancyhdr}
\pagestyle{fancy}
\usepackage{lastpage}
\makeatletter

\lhead{\textbf{\@title Tutor:} Paul Podlech \\ \@author}
\chead{}
\rhead{\@date \\ \thepage \ von \pageref{LastPage} }
\cfoot{}
%\cfoot{\small \textbf{Disclaimer:} Einige Lösungen wurden mit einer anderen Übungsgruppe (Jens Fischer, Johannes Dillmann, Tobias Famulla) inhaltlich diskutiert,  eine gewisse Ähnlichkeit der Lösungen ist möglich. Trotzdem sind alle Lösungen selbstständig von den hier genannten Mitgliedern erarbeitet.}
\renewcommand{\labelenumi}{\alph{enumi})}
\renewcommand{\maketitle}{}
\newcommand{\utilde}[1]{\underaccent{\tilde}{#1}}
\renewcommand{\familydefault}{\sfdefault}

\author{Florian Ritzel, Hinnerk van Bruinehsen, Tobias Höppner}
\title{SvP - Übung 07. }
\date{Sommersemester 2014}
\begin{document}
\maketitle
\section*{Aufgabe 1}
Sei $\underline{A} = (A,\sqsubseteq)$ eine Struktur, wobei die Relation $\sqsubseteq$ eine Halbordnung ist und es in $A$ bzgl. $\sqsubseteq$ ein minimales Element $\perp$ gibt.\\
Zeigen Sie, dass aus der Endlichkeit von $A$ folgt, dass $\underline{A}$ ein cpo ist.\\

%\textbf{Lösungsvorschlag Hinnerk:}\\
Es müssen drei Bedingungen erfüllt sein, damit eine Struktur $A$ ein cpo ist:
\begin{enumerate}
\item $\sqsubseteq$ ist Halbordnung.
\item In $A$ existiert ein minimales Element.
\item Zu jeder Kette $K\subseteq A$ existiert eine kleinste obere Schranke.
\end{enumerate}
Aus der Aufgabenstellung folgt, dass Bedingung 1 ($\sqsubseteq$ ist Halbordnungsrelation) und 2 (es gibt minimales Elemement $\bot$) erfüllt sind.
Um zu beweisen, dass $A$ ein cpo ist, müssen wir also zeigen, dass zu jeder Kette $K\sqsubseteq A$ eine kleinste obere Schranke existiert.
Aus der Endlichkeit von A folgt, dass es ein Element $\top$ gibt, für das gilt: $\top \geq x$ für alle $x \in A$. Aus der Definition der oberen Schranke folgt, dass es eine obere Schranke gibt, wenn ein größtes Element existiert. Daraus folgt wiederum, dass es eine kleinste obere Schranke für jede Kette $K\subseteq A$ geben muss.

Die Bedinungen 1 und 2 sind entsprechend der Aufgabenstellung erfüllt. Daher ist zu zeigen, dass aus der Endlichkeit von $A$ folgt, dass zu jeder kette $K\subseteq A$ eine kleinste obere Schanke existiert.\\

Aus der Endlichkeit von A folgt, dass es ein Element $\top$ gibt, dass $\geq$ aller anderen Elemente ist. Aus der Definition der oberen Schranke folgt, dass es, wenn es ein größtes Elemement gibt, auch eine obere Schranke gibt. Daraus folgt, dass es eine kleinste obere Schranke für jede Kette $K\subseteq A$ geben muss
\section*{Aufgabe 2}
Gegeben sei eine halbgeordnete Menge $A$, die sich grafisch wie folgt darstellen lässt:\\
\begin{figure}[h]
\centering
\begin{tikzpicture}[-,>=stealth',auto,node distance=2cm,
  thick,main node/.style={circle, fill=black, minimum size=4pt, inner sep=.1pt, outer sep=.1pt}]
  \node[] (11){};
  \node[main node] (1) [left of=11] {};
  \node[main node] (2) [right of=11] {};
  \node[main node] (4) [below of=11] {};
  \node[] (41)[left of=4]{};
  \node[] (42)[right of=4]{};
  \node[main node] (3) [left of=41] {};
  \node[main node] (5) [right of=42] {};
  \node[main node] (6) [below of=4] {};
  \path[every node/.style={font=\sffamily\small}]
    (1) edge node {} (3)
    (1) edge node {} (4)
    (2) edge node {} (4)
    (2) edge node {} (5)
    (4) edge node {} (6)
    (3) edge node {} (6)
    (5) edge node {} (6);
\end{tikzpicture}

\end{figure}
\begin{compactitem}
\item[a)] Ist $A$ ein cpo?\\
%\textbf{Lösungsvorschlag Hinnerk:}\\
	Ja, alle Kritierien sind erfüllt.
\item[b)] Ist $A$ eine Kette?\\
%\textbf{Lösungsvorschlag Hinnerk:}\\
	Nein, die mittleren drei Elemente stehen nicht in Relation zu einander, die oberen zwei ebenso wenig.
\item[c)] Existiert eine kleinste obere Schranke von $A$ in $A$?\\
%\textbf{Lösungsvorschlag Hinnerk:}\\
	Nein, da die beiden oberen Elemente augenscheinlich auf einer Höhe liegen.
\end{compactitem}
\newpage
\section*{Aufgabe 3}
Finden Sie zwei Beispiele für Halbordnungen $(A,\sqsubseteq)$, die ein minimales Element besitzen aber keine cpo's sind.

%\textbf{Lösungsvorschlag Hinnerk:}\\
\begin{enumerate}
	\item $f(x) = \sqrt{x}$ für $x \in \mathbb{R}$ (Besitzt minimales Element $0$, ist aber nach oben nicht beschränkt.
	\item $f(x) = x$ für $x \in \mathbb{N}$ (wie oben).
\end{enumerate}
\section*{Aufgabe 4}
Finden Sie ein Beispiel für eine nicht triviale, rekursive Funktionsgleichung, die mehr als eine Lösung hat.

%\textbf{Lösungsvorschlag Hinnerk:}\\
$f(x) = \left\lbrace\begin{array}{l l}
	x, & \text{ wenn } f(x) \mod 2 = x \\
	0, & \text{ wenn } f(x) \mod 2 \neq x
	\end{array}\right.$


Ein Klassiker: Fibonacci.\\
$g(x) = \left\lbrace\begin{array}{l l}
	0, & \text{ falls } x = 0 \\
	1, & \text{ falls } x = 1 \\
	g(x-1) + g(x-2), & \text{ falls } x = 1 \\
	\end{array}\right.$	
%Macht das Sinn? Ist das zu Trivial?
%(Für 1 ist es 1, für alle anderen Werte ist es 0)
\end{document}




