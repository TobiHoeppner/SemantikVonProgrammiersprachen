\documentclass[ngerman,a4paper]{report}
\usepackage[ngerman]{babel}
\usepackage[T1]{fontenc}
\usepackage[utf8]{inputenc}
\usepackage{lmodern}
\usepackage{MyriadPro}
\usepackage[scaled]{beramono}
\newcommand\Small{\fontsize{10.5}{10.5}\selectfont}
\newcommand*\LSTfont{\Small\ttfamily\SetTracking{encoding=*}{-20}\lsstyle}
\usepackage{microtype}
\usepackage{geometry}
\geometry{verbose,tmargin=3cm,bmargin=3cm,lmargin=3cm,rmargin=3cm}
\usepackage{centernot}
\usepackage{listings}
\usepackage{paralist}
%\usepackage{array}
\usepackage{xcolor}
%\usepackage{graphicx}
%\usepackage{caption}
%\usepackage{url}
\usepackage[verification]{struktex}
\usepackage{amsmath}
\usepackage{mathtools}
%\usepackage{accents}
%\usepackage{tikz}
\usepackage{stmaryrd}

% Code Listing Style
\definecolor{darkblue}{rgb}{0,0,.6}
\definecolor{darkgreen}{rgb}{0,0.5,0}
\definecolor{darkred}{rgb}{0.5,0,0}
\lstset{%
	language=C,
	basicstyle=\LSTfont,
	commentstyle=\color{darkgreen},
	keywordstyle=\color{darkblue}\bfseries,
	breaklines=true,
	tabsize=2,
	xleftmargin=\fboxsep,
	xrightmargin=-\fboxsep,
	numbers=left,
	numberstyle=\tiny\color{gray},
	stepnumber=1,
	numbersep=5pt,
	frame=bt,
	stringstyle=\color{darkred},
	showstringspaces=false,
	rulecolor= \color{gray},
	emph=[1]%
	{%
	    then, not%
	},
	emphstyle=[1]{\color{darkblue}\bfseries},
	emph=[2]%
	{%  Datatypes
	    %
	},
	emphstyle=[2]{\color{darkblue}\bfseries},
	emph=[3]%
	{%
	    %
	},
	emphstyle=[3]{\color{darkred}\bfseries},
	literate=%
	{Ö}{{\"O}}1
	{Ä}{{\"A}}1
	{Ü}{{\"U}}1
	{ß}{{\ss}}2
	{ü}{{\"u}}1
	{ä}{{\"a}}1
	{ö}{{\"o}}1
}
\providecommand{\tabularnewline}{\\}

\usepackage{fancyhdr}
\pagestyle{fancy}
\usepackage{lastpage}
\makeatletter

\lhead{\textbf{\@title Tutor:} Paul Podlech \\ \@author}
\chead{}
\rhead{\@date \\ \thepage \ von \pageref{LastPage} }
\cfoot{}
%\cfoot{\small \textbf{Disclaimer:} Einige Lösungen wurden mit einer anderen Übungsgruppe (Jens Fischer, Johannes Dillmann, Tobias Famulla) inhaltlich diskutiert,  eine gewisse Ähnlichkeit der Lösungen ist möglich. Trotzdem sind alle Lösungen selbstständig von den hier genannten Mitgliedern erarbeitet.}
\renewcommand{\labelenumi}{\alph{enumi})}
\renewcommand{\maketitle}{}
\newcommand{\utilde}[1]{\underaccent{\tilde}{#1}}
\renewcommand{\familydefault}{\sfdefault}

\author{Florian Ritzel, Hinnerk van Bruinehsen, Tobias Höppner}
\title{SvP - Übung 06. }
\date{Sommersemester 2014}
\begin{document}
\maketitle
\section*{Aufgabe 1}
Beweisen Sie die Formel
\begin{align*}
\{true\}\quad
x:=7;y:=x+3\quad
\{y=10\}
\end{align*}
im Hoare-Kalkül.\\
%\textbf{Idee von Tobi:}\\
%kurze Regelkunde:
% \frac{\{P\}S\{R\}\{R\}T\{Q\}}{\{P\}S;T\{Q\}}\tag{Sequenz}\\
%\begin{align}
%\{P[x\leftarrow e]	\}\quad	x:=e 	\quad						\{P\}\tag{Zuweisung}\\
%\notag
%\end{align}
%Mit anderen Worten: Man ersetzt in der Nachbedingung alle vorkommen von x durch e.\\
%and here we go: (von hinten nach vorne, also von unten nach oben lesen ;) )
%\begin{align*}
%\{true\}\quad & 			\quad &\{true\} \\
%\{true\}\quad &			\quad &\{7+3=10\} \tag{Zuweisung}\\
%\{true\}\quad &	x:=7		\quad &\{x+3=10\} \tag{Zuweisung}\\
%\{true\}\quad &	x:=7;y:=x+3		\quad &\{y=10\}\\
%\end{align*}
%
%\textbf{Struktex Version (Hinnerk):}\\
\begin{struktogramm}(50,25)
\assert{$\{P\} true$}
\assert{$7=7 \Rightarrow true$}
\assign{$x:=7$}
\assert{$x+3=10 \Rightarrow x=7$}
\assign{$y:=x+3$}
\assert{$\{Q\} y=10$ }
\end{struktogramm}\\
\section*{Aufgabe 2}
Schreiben Sie ein WHILE'-Programm zur Berechnung der Signum-Funktion und beweisen Sie seine Korrektheit im Hoare-Kalkül.\\
%\textbf{Idee von Tobi:}\\ % Anmerkung Tobi: DAS IST MUELL!!! NICHT VERWENDEN, ICH HAB MICH VERLESEN xD!
%Erstmal die Regeln
%\begin{align}
%\frac{\{I\land B\}\quad S\quad\{I\}}{\{I\}\quad \underline{while}\;B\;\underline{do}\;S\quad\{I\land \lnot B\}}\tag{while}\\
%\notag
%\end{align}
%\begin{lstlisting}
%sum:=0;
%while not eof do
%	read x;
%	sum := sum + x;
%output sum
%\end{lstlisting}
%Die Vorbedingung $P$ ist relativ egal. Viel wichtiger ist die Nachbedingung $Q$, die muss den Zustand $in=\epsilon$ enthalten, da das Programm terminieren muss und der Wahrheitswert von $\underline{eof}$ davon abhängt. Im folgenden werden die Variablen $out$ für die Ausgabe und $in$ für die Eingabe verwendet. Das Symbol $\epsilon$ stellt hier eine leere Menge dar.\\
%Die Invariante für die Whileschleife sollte eventuell nur Festhalten, dass die Ausgabe unverändert bleibt, die Schleifenbedingung lässt sich wie folgt definieren:\\ $B = \{\lnot(in = \epsilon) \} = \{in \neq \epsilon\} = \{in = [n_0,\dots,n_i]\}$
%\begin{align*}
%P = &\{True\}																						\\
%\{\} \quad		&sum:=0; 		\;\;\quad\quad\quad\quad\quad	\{out = \epsilon \land in \neq \epsilon \land 0 + (n_0) \neq \epsilon\}				\\
%\{\} \quad 		&\underline{while}\; \lnot \underline{eof}\; \underline{do}\;		\;\;\quad\quad\quad	\{out = \epsilon \land in \neq \epsilon \land sum + (n_0) \neq \epsilon\}	\\
%\{\} \quad 		&\quad	\underline{read}\; x;	\ \;\;\quad\quad\quad\quad\quad	\{out = \epsilon \land in = [n_1, \dots, n_i] \land sum + (n_0) \neq \epsilon\}\\
%\{\} \quad 		&\quad	sum := sum + x;												\;\quad\{out = \epsilon \land in \neq \epsilon \land sum + x \neq \epsilon\}\\
%\{\} \quad 		&\underline{output}\; sum							\quad\quad\quad\quad\quad\{out = \epsilon \land in = \epsilon \land sum \neq \epsilon \}	\\
%Q = &\{out = sum \land in = \epsilon \}																\\
%\end{align*}

\begin{lstlisting}
read x
signum := 0
if x > 0 then 
	signum := 1
if x < 0 then
	signum := -1
\end{lstlisting}

%\textbf{Struktex Version (Hinnerk):}\\
\begin{struktogramm}(150,120)
\assert{$\{I\} x\in\mathbb{Z}$}
\assign{\lstinline!read x!}
\assert{$\{I\} x\in\mathbb{Z} \wedge (0=1\vee0=-1\vee0=0)$}
\assign{\lstinline!signum := 0!}
\assert{$\{I\} x\in\mathbb{Z}\wedge(signum=1\vee signum=-1\vee signum=0)$}
\ifthenelse[10]{3}{2}{\lstinline!if x > 0!}{\lstinline!true!}{\lstinline!false!}
\assert{ }
\assert{$\{I\} x\in \mathbb{Z} \wedge x > 0 \ true$}
\assert{$\{I\} (1=1)\vee (1=-1)\vee (1=0) \Rightarrow true \vee false\vee false$}
\assign{\lstinline!signum := 1!}
\assert{$\{I\} x\in\mathbb{Z}\wedge(signum=1\vee signum=-1\vee signum=0)$}
\change
\assert{$\{I\}x \in \mathbb{Z} \wedge x\leq0 \wedge (signum=1\vee signum=-1\vee signum=0)$}
\assign[20]{\lstinline!skip!}
\assert{$\{I\} x\in\mathbb{Z}\wedge (signum=1\vee signum=-1\vee signum=0)$}
\ifend
\assert{$\{I\} x\in\mathbb{Z}\wedge(signum=1\vee signum=-1\vee signum=0)$}
\ifthenelse[10]{3}{2}{\lstinline!if x < 0!}{\lstinline!true!}{\lstinline!false!}
\assert{ }
\assert{$\{I\} x\in \mathbb{Z} \wedge x < 0 \ true$}
\assert{$\{I\} (-1=1)\vee (-1=-1)\vee (-1=0) \Rightarrow false \vee  true\vee false$}
\assign{\lstinline!signum := -1!}
\assert{$\{I\} x\in\mathbb{Z}\wedge(signum=1\vee signum=-1\vee signum=0)$}
\change
\assert{$\{I\}x \in \mathbb{Z} \wedge x\geq0 \wedge (signum=1\vee signum=-1\vee signum=0)$}
\assign[20]{\lstinline!skip!}
\assert{$\{I\} x\in\mathbb{Z}\wedge (signum=1\vee signum=-1\vee signum=0)$}
\ifend
\assert{$\{I\} x\in\mathbb{Z}\wedge(signum=1\vee signum=-1\vee signum=0)$}
\end{struktogramm}
\newpage
\section*{Aufgabe 3}
Führen Sie einen Korrektheitsbeweis unter Verwendung der axiomatischen Semantik zu folgendem Programm:
\begin{lstlisting}
sum:=0;
while not eof do
	read x;
	sum := sum + x;
output sum
\end{lstlisting}
%\textbf{Hinnerks Idee:}\\
\begin{struktogramm}(100,90)
\assert{$\{I\} sum\in\mathbb{Z}$}
\assign{\lstinline!sum := 0!}
\assert{$\{I\} sum\in\mathbb{Z}$}
\while[12]{$\neg eof$}
  \assert{$\{I\} sum\in\mathbb{Z}\wedge \neg eof$}
  \assign{\lstinline!read x!}
  \assert{$\{I\} sum\in\mathbb{Z}$}
  \assign{\lstinline!sum := sum + x!}
  \assert{$\{I\} sum\in\mathbb{Z}$}
\whileend
\assert{$\{I\wedge \neg B\} sum\in\mathbb{Z}\wedge eof$}
\assign{\lstinline!output sum!}
\assert{$\{I\} sum\in\mathbb{Z}$}
\end{struktogramm}\\
\section*{Aufgabe 4}
Beweisen Sie die Gültigkeit des Axioms (A.4), d.h. zeigen Sie die Gültigkeit der Formel:
\begin{align*}
\{Q[output.T/output]\}\quad
output T\quad
\{Q\}
\end{align*}

Wir unterscheiden zwei Fälle:
\begin{enumerate}
	\item \textbf{Fehler:}\\
\begin{struktogramm}(100,40)
\assert{$\{Q[output.T/output]\}$}
\assert{$\{Q[output.T/output]\} Fehler$}
\assign{\lstinline!output T!}
\assert{$\{Q\} Fehler$}
\end{struktogramm}
\item \textbf{kein Fehler:}\\
\begin{struktogramm}(100,20)
\assert{$\{Q[output.T/output]\}$}
\assert{$\{Q[output.T/output]\}(n,(s,e',a)) $}
\assign{\lstinline!output T!}
\assert{$\{Q\}(s,e',a.n)$}
\end{struktogramm}
\end{enumerate}
\end{document}
