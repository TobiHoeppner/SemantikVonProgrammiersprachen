\documentclass[ngerman,a4paper]{report}
\usepackage[ngerman]{babel}
\usepackage[T1]{fontenc}
\usepackage[utf8]{inputenc}
\usepackage{MyriadPro}
\usepackage[scaled]{beramono}
\newcommand\Small{\fontsize{10.5}{10.5}\selectfont}
\newcommand*\LSTfont{\Small\ttfamily\SetTracking{encoding=*}{-20}\lsstyle}
\usepackage{microtype}
\usepackage{geometry}
\geometry{verbose,tmargin=3cm,bmargin=3cm,lmargin=3cm,rmargin=3cm}
\usepackage{centernot}
\usepackage{listings}
%\usepackage{ stmaryrd }
\usepackage{mathtools}
\usepackage{paralist}
\usepackage{array}
\usepackage{color}
\usepackage{graphicx}
\usepackage{caption}
\usepackage{url}
\usepackage{amsmath}
\usepackage{accents}
\usepackage{tikz}

% Code Listing Style
\definecolor{darkblue}{rgb}{0,0,.6}
\definecolor{darkgreen}{rgb}{0,0.5,0}
\definecolor{darkred}{rgb}{0.5,0,0}
\lstset{%
	language=C,
	basicstyle=\LSTfont,
	commentstyle=\color{darkgreen},
	keywordstyle=\color{darkblue}\bfseries,
	breaklines=true,
	tabsize=2,
	xleftmargin=\fboxsep,
	xrightmargin=-\fboxsep,
	numbers=left,
	numberstyle=\tiny\color{gray},
	stepnumber=1,
	numbersep=5pt,
	frame=bt,
	stringstyle=\color{darkred},
	showstringspaces=false,
	rulecolor= \color{gray},
	emph=[1]%
	{%
	    then, not%
	},
	emphstyle=[1]{\color{darkblue}\bfseries},
	emph=[2]%
	{%  Datatypes
	    %
	},
	emphstyle=[2]{\color{darkblue}\bfseries},
	emph=[3]%
	{%
	    %
	},
	emphstyle=[3]{\color{darkred}\bfseries},
	literate=%
	{Ö}{{\"O}}1
	{Ä}{{\"A}}1
	{Ü}{{\"U}}1
	{ß}{{\ss}}2
	{ü}{{\"u}}1
	{ä}{{\"a}}1
	{ö}{{\"o}}1
}
\providecommand{\tabularnewline}{\\}

\usepackage{fancyhdr}
\pagestyle{fancy}
\usepackage{lastpage}
\makeatletter

\lhead{\textbf{\@title Tutor:} Paul Podlech \\ \@author}
\chead{}
\rhead{\@date \\ \thepage \ von \pageref{LastPage} }
\cfoot{}
%\cfoot{\small \textbf{Disclaimer:} Einige Lösungen wurden mit einer anderen Übungsgruppe (Jens Fischer, Johannes Dillmann, Tobias Famulla) inhaltlich diskutiert,  eine gewisse Ähnlichkeit der Lösungen ist möglich. Trotzdem sind alle Lösungen selbstständig von den hier genannten Mitgliedern erarbeitet.}
\renewcommand{\labelenumi}{\alph{enumi})}
\renewcommand{\maketitle}{}
\newcommand{\utilde}[1]{\underaccent{\tilde}{#1}}
\renewcommand{\familydefault}{\sfdefault}

\author{Florian Ritzel, Hinnerk van Bruinehsen, Tobias Höppner}
\title{SvP - Übung 04. }
\date{Sommersemester 2014}
\begin{document}
\maketitle
\section*{Aufgabe 1}
Zeigen Sie für folgendes Programm $P$
\begin{lstlisting}
x := 5; y := 2; output (x - (y + read))
\end{lstlisting}
dass sowohl die operationelle Semantik als auch die Reduktionssemantik bei Eingabe $E = (4)$ die Ausgabe $A = (-1)$ bestimmt.
\section*{Aufgabe 2}
Gegeben sei folgende Syntax:
\begin{lstlisting}
W := True | False
LOP := AND | OR
LA := W | LA1 LOP LA2 | Not LA
\end{lstlisting}
zur Formalisierung logischer Ausdrücke.
\begin{enumerate}
\item Definieren Sie eine geeignete operationelle Semantik.
\item Definieren Sie eine geeignete Reduktionssemantik.
\item Beweisen Sie die Äquivalenz Ihrer Lösungen zu a) und b).
\end{enumerate}
\section*{Aufgabe 3 \text{(freiwillig)}}
\begin{enumerate}
	\item Implementieren Sie die Reduktionssemantik von WHILE in eine Programmiersprache Ihrer Wahl.
	\item Implementieren Sie die Semantikfunktion eval, die jeder Programm-Daten-Kombination die entsprechende Ausgabe zuordnet.
	\item Testen Sie Ihre Funktion eval am Beispiel des ganzahligen Divisionsprogramms.
\end{enumerate}
\textbf{Hinweis:} Bei Besprechung dieser Aufgabe wird ein Beamer zur Verfügung stehen.
\end{document}
