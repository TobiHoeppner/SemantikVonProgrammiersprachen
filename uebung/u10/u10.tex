\documentclass[ngerman,a4paper]{report}
\usepackage[ngerman]{babel}
\usepackage[T1]{fontenc}
\usepackage[utf8]{inputenc}
\usepackage{lmodern}
\usepackage{MyriadPro}
\usepackage[scaled]{beramono}
\newcommand\Small{\fontsize{10.5}{10.5}\selectfont}
\newcommand*\LSTfont{\Small\ttfamily\SetTracking{encoding=*}{-20}\lsstyle}
\usepackage{microtype}
\usepackage{geometry}
\geometry{verbose,tmargin=3cm,bmargin=3cm,lmargin=3cm,rmargin=3cm}
\usepackage{centernot}
\usepackage{listings}
\usepackage{paralist}
%\usepackage{array}
\usepackage{xcolor}
%\usepackage{graphicx}
%\usepackage{caption}
%\usepackage{url}
%\usepackage[verification]{struktex}
\usepackage{amsmath}
\usepackage{amsfonts}
\usepackage{mathtools}
%\usepackage{accents}
\usepackage{tikz}
\usetikzlibrary{shapes,arrows,automata}
\tikzset{
	treenode/.style = {align=center, inner sep=1pt, text centered,font=\sffamily},
	node/.style = {treenode, font=\sffamily\bfseries, text width=1.5em},
	cloud/.style = {draw, circle, fill=red!20, node distance=3cm, minimum height=2em},
	decision/.style = {diamond, draw, fill=blue!20, text width=6em, text badly centered, node distance=3cm, inner sep=0pt},
	block/.style = {rectangle, draw, fill=green!20, text width=6em, text centered, rounded corners, minimum height=4em},
	line/.style = {draw, -latex'}
}
\usepackage{stmaryrd}

% Code Listing Style
\definecolor{darkblue}{rgb}{0,0,.6}
\definecolor{darkgreen}{rgb}{0,0.5,0}
\definecolor{darkred}{rgb}{0.5,0,0}
\lstset{%
	language=C,
	basicstyle=\LSTfont,
	commentstyle=\color{darkgreen},
	keywordstyle=\color{darkblue}\bfseries,
	breaklines=true,
	tabsize=2,
	xleftmargin=\fboxsep,
	xrightmargin=-\fboxsep,
	numbers=left,
	numberstyle=\tiny\color{gray},
	stepnumber=1,
	numbersep=5pt,
	frame=bt,
	stringstyle=\color{darkred},
	showstringspaces=false,
	rulecolor= \color{gray},
	emph=[1]%
	{%
	    then, not%
	},
	emphstyle=[1]{\color{darkblue}\bfseries},
	emph=[2]%
	{%  Datatypes
	    %
	},
	emphstyle=[2]{\color{darkblue}\bfseries},
	emph=[3]%
	{%
	    %
	},
	emphstyle=[3]{\color{darkred}\bfseries},
	literate=%
	{Ö}{{\"O}}1
	{Ä}{{\"A}}1
	{Ü}{{\"U}}1
	{ß}{{\ss}}2
	{ü}{{\"u}}1
	{ä}{{\"a}}1
	{ö}{{\"o}}1
}
\providecommand{\tabularnewline}{\\}

\usepackage{fancyhdr}
\pagestyle{fancy}
\usepackage{lastpage}
\makeatletter

\lhead{\textbf{\@title Tutor:} Paul Podlech \\ \@author}
\chead{}
\rhead{\@date \\ \thepage \ von \pageref{LastPage} }
\cfoot{}
%\cfoot{\small \textbf{Disclaimer:} Einige Lösungen wurden mit einer anderen Übungsgruppe (Jens Fischer, Johannes Dillmann, Tobias Famulla) inhaltlich diskutiert,  eine gewisse Ähnlichkeit der Lösungen ist möglich. Trotzdem sind alle Lösungen selbstständig von den hier genannten Mitgliedern erarbeitet.}
\renewcommand{\labelenumi}{\alph{enumi})}
\renewcommand{\maketitle}{}
\newcommand{\utilde}[1]{\underaccent{\tilde}{#1}}
\renewcommand{\familydefault}{\sfdefault}

\author{Florian Ritzel, Hinnerk van Bruinehsen, Tobias Höppner}
\title{Anonyme Semantiker - Übung 10. }
\date{Sommersemester 2014}
\begin{document}
\maketitle
\section*{Aufgabe 1}
Bestimmen Sie die Typen folgender Funktionen:
\begin{itemize}
\item[(i)]$\lambda fx.(fx)+1$\\
Lösung aus dem Tutorium:
\begin{align*}
	1:& \mathbb{Z}_\perp\\
	+:& [\mathbb{Z}_\perp \times \mathbb{Z}_\perp \rightarrow \mathbb{Z}_\perp]\\
	x:& D\\
	f:& [D \rightarrow \mathbb{Z}_\perp]\\
\end{align*}
alternativ, allgemeiner
\begin{align*}
	1:& D_1\\
	+:& [D_2 \times D_1 \rightarrow D_3]\\
	f:& [D_4 \rightarrow D_2]\\
	x:& D_4\\
\end{align*}
\item[(ii)] $\lambda(x,y)f.fxy$\\
Lösung aus dem Tutorium:
\begin{align*}
	x :& D_1\\
	y :& D_2\\
	f :& D_1 \rightarrow D_2 \rightarrow D_3\\
	:& (D_1 \times D_2) \rightarrow D \rightarrow D_3\\
\end{align*}
\item[(iii)] $\lambda f.(f\lambda y.y)$
Lösung aus dem Tutorium:
\begin{align*}
	y :& D_1\\
	id :&= \lambda y.y : D_1 \rightarrow D_1\\
	f :& [D_1 \rightarrow D_1] \rightarrow D_2\\
\end{align*}
\end{itemize}

\section*{Aufgabe 2}
Der Faltungsoperator $\underline{lit}$ sei informell bestimmt durch:
$\underline{lit} f<x_1,\dots,x_n>x_{n+1} = fx_1(fx_2(\dots(fx_n x_{n+1})\dots))$\\
z.B. $\underline{lit} \ \underline{plusc}<x_1,\dots,x_n>x_{n+1} = x_1 + x_2 + \dots + x_{n+1}$\\
\begin{itemize}
\item[(i)] Bestimmen Sie den Typ von $\underline{lit}$\\
Aus dem Tut:
\begin{align*}
\underline{lit}: [D_1 \rightarrow D_2 \rightarrow D_2] \rightarrow D_1^* \rightarrow D_2 \rightarrow D_2\\
\end{align*}
\item[(ii)] Definieren Sie den Operator $\underline{lit}$ im getypten $\lambda$-Kalkül unter Verwendung der Gleichungsschreibweise (s. S. 102).\\
Aus dem Tut:
\begin{align*}
	(a)&& \underline{lit} &:= \lambda	 f L x. (L = <> \rightarrow x), f (\underline{hd}\ L) (\underline{lit}\ f(\underline{tl}\ x))\\
	(b)&& \underline{lit}	 f L x &:= L = <> \rightarrow x, f (\underline{hd}\ L) (\underline{lit}\ f(\underline{tl}\ x))\\
\end{align*}
\item[(iii)] Definieren Sie eine Funktion $f$ im getypten $\lambda$-Kalkül, so dass\\
$
f<x_1,\dots,x_n>x=\begin{cases}
\text{wahr, falls } x=x_i \text{ für ein } i\\
\text{falsch, sonst.}
\end{cases}
$\\
Aus dem Tut:
\begin{align*}
	f :& D_1^* \rightarrow D_1 \rightarrow \text{\lstinline!BOOL!}\\
	\underline{eq} :& D_1 \rightarrow D_1 \rightarrow \text{\lstinline!BOOL!}\\
				 &\underline{eq}\ d_1 \ d_2 = \begin{cases} \underline{true}, &\text{falls } d_1 = d_2 \\ \underline{false}, &\text{sonst}
\end{cases}\\
&f := \lambda L\ d.\underline{lit}(\lambda x. \underline{or} (\underline{eq}\ d\ x)) L \underline{false}\\
\end{align*}
\item[(iv)] Bearbeiten Sie (i)-(iii) für $\underline{lit'}\ fx_1<x_2,\dots,x_{n+1}>=(f\dots(f(fx_1 x_2)x_3)\dots x_{n+1})$
\end{itemize}

\section*{Aufgabe 3}
Erweitern Sie die Syntax von WHILE um Anweisungen der Form\\
$\underline{repeat}\ C\ \underline{until}\ B$\\
und definieren Sie dazu eine geeignete denotationelle Semantik.

\begin{align*}
	(a)&& \mathcal{C}\llbracket \underline{repeat}\ C_1\ \underline{until}\ B\rrbracket\\
	&&&= \mathcal{C}\llbracket C_1; \underline{while}\ \underline{not}\ B\ \underline{do}\ C_1\rrbracket\\
\\
(b)&&\mathcal{C}\llbracket\underline{repeat}\ C_1\ \underline{until}\ B\rrbracket\\
   & &&= \mathcal{C} \llbracket C_1\rrbracket\bowtie\mathcal{B}\llbracket B\rrbracket \bowtie cond <\lambda x.x, \mathcal{C}\llbracket C'\rrbracket > b
\end{align*}


\end{document}
