\section{Axiomatische Semantik \small (Vorlesung 6 am 05.06.)}
% höhe von Grundlage
\marginnote{\small\emph{(Hoare-Kalkül zum Bew. von Prog. Eigenschft.)}}[0,5cm]
% höhe Allgemeine Methode 1.)
\marginnote{\small\emph{(formales Setting von logischen Ausdrücken vorgeben, damit man damit arbeiten kann!)}}[3cm]

Am Beispiel von WHILE' ($\underline{read}\;I$ anstelle von Termen der Form $\underline{read}$).\\
\textbf{Grundlage:} C. A. R. Hoare: \emph{An aximatic Basis for Computer Programming} (CACM 1969)
\subsection{Allgemeine Methode}
\textbf{Annahme:} AST ist gegeben.\\
Neben der abstrakten Syntax brauchen wir vier Angaben:
\begin{compactitem}
	\item[\textbf{1.}] Eine Menge von Bedingungen (\emph{logische Ausdrücke, Prädikate, assertions}), die über den Zustandsraum interpretierbar sind. Im Allgemeinen Prädikatenlogik 1. Stufe oder Aussagenlogik.\\
	(formales Setting von logischen Ausdrücken vorgeben, damit man damit arbeiten kann!)
	Beispiel: $x < y, input = \epsilon, \exists i.x=p.i, ...$ 
	natürliche Interpretation: Verbinde $I \in$ ID mit $S(I)$ sowie $\underline{input}$ mit Eingabe....\\
	Sei $S(x)=0$, dann ist die Bedingung $y/x=7$ \underline{falsch}.\\
	\item[\textbf{2.}] Definition zu jeder atomaren Anweisung $\mathcal{C}$ ein Axiom bzw. ein Axiomenschema der Form\\
	$\{Q\}C\{R\}$ - Hoare Formel\\
	zu lesen als: Wenn die Bedingung $Q$ auf einem Zustand $z$ gilt und die Ausführung von $C$ auf $z$ in einem Zustand $z'$ terminiert, dann gilt $R$ auf $z'$.\\
	Beispiel: für $\underline{skip}$ des Axiomenschema: $\{Q\}\;\underline{skip}\;\{Q\}$, \emph{zu lesen: für alle $Q$ gilt es}.\\
	\item[\textbf{3.}] Für zusammengesetzte Anweisungen $C$ mit unmittelbaren Komponenten $C_1,\dots,C_r$ eine Schlussregel der Form $\frac{F_1,\dots,F_i}{\{Q\}C\{R\}}$, wobei die $F_i$ Hoare-Formeln zu $C1, \dots, C_r$ oder andere Formeln sind.\\
	\item[\textbf{4.}] \emph{Logischer Klebstoff}: Allgemeine Schlussregeln in Verbindung mit Hoare-Formeln, mindestens (oft ausreichend) die Konsequenzregel: $\frac{Q\Rightarrow S, T \Rightarrow R, \{S\}C\{T\}}{\{Q\}C\{R\}}$, wobei $\Rightarrow$ die logische Herleitung bezeichnet. Daraus folgt: Beweisbar $\Rightarrow$ Gültig. Frage: gilt diese Herleitung auch andersrum ($F$ gültig $\Rightarrow$ beweisbar)? Nein. Im Allgemeinen landet man beim Haltemaschinenproblem und dem Gödelschen Unvollständigkeitssatz.\\
\end{compactitem}
\subsection{Konkretisierung am Beispiel von WHILE'}
%TODO als Einschub:
\begin{align*}
T&::= Z\; |\; I\; |\; T_1\; \underline{OP}\; T_2\\
B&::= \underline{true}\; |\; \underline{false}\; |\; \underline{eof}\; |\; T_1\; \underline{Relop}\; T_2\\
C&::= \underline{skip}\; |\; \underline{read}\; I\; |\; I:=T\; |\; \underline{output}\; T\; |\; C_1;C_2\; |\; \underline{if}\;B\;\underline{then}\;C_1\;\underline{else}\;C_2\;|\;\underline{while}\;B\;\underline{do}\;C
\end{align*}
% höhe Formel A.1
\begin{compactitem}
	\item[\textbf{1.}] Aussagenlogik:
	\begin{itemize}
		\item Terme über $Z, I, OP, \underline{input}, \underline{output},$ Konstruktor- und Selektorfunktionen ($., \underline{hd}, \underline{tl}$)
		\item Bedingungen: $T_1\;\underline{Relop}\;T_2$\\
		$B_1 \lor B_2, B_1 \land B_2, \lnot B$ 
	\end{itemize}
\newpage
\marginnote{\small\emph{(zu lesen: Für jede Bedingung $Q$ gilt die Hoare-Formel $\{Q\}\underline{skip}\{Q\}$)}}[1cm]
	\item[\textbf{2.}] atomare Definitionen:\\
		\begin{align}
			\{Q\}&\underline{skip}\{Q\}\tag{A.1}\\
			\{Q[T/I]\}&I:=T\{Q\},\notag\\&\text{wobei}[T/I]\text{ die einmalige textliche Substitution von} \notag\\&T \text{ für jedes Vorkommen von I in Q bezeichnet.}\tag{A.2}\\
			\{Q[\underline{hd} \; \underline{input}/I\; |\; \underline{tl} input / \underline{input}]\;\underline{read}\; I \{Q\} &\approx I:= \underline{hd}(\underline{
			input}); \underline{input} := \underline{tl}\; (\underline{input}) \tag{A.3}\\
			\{Q[output.T / \underline{output}]\}\;\underline{output}\;T \{Q\} &\approx \underline{output}=\underline{output}.T\tag{A.4}\\
			\notag
		\end{align}
		Beispiele:
		\begin{align*}
		\{input=(4,5)\land2+x=3\} I:=2+x \{input=(4,5)\land I=3\}
		\end{align*}
		\begin{align*}
		\{\underline{hd}\;\underline{input}=7 \land \underline{tl}\; \underline{tl}\;\underline{input}=(3) \}\;\underline{read}\; x \{x=7 \land input=(3)\}
		\end{align*}
		\begin{align*}
		\{\underline{output}.7+y=(1,2,3,4,5)\}\; \underline{output} (7+9) \{\underline{output} = (1,2,3,4,5)\}
		\end{align*}
	\item[\textbf{3.}]
	\begin{align}
	\frac{\{Q\}C_1\{S\},\{S\}C_2\{R\}}{\{Q\}C_1,C_2\{R\}}\tag{A.5}\\
	\frac{\{Q\land B\}C_1\{R\},\{Q\land \lnot B\}C_2\{R\}}{\{Q\}\underline{if}B\underline{then}C_1\underline{else}C_2\{R\}}\tag{A.6}\\
	\frac{\{Q\land B\}C\{Q\}}{\{Q\}\underline{while}\;B\;\underline{do}\;C\{Q\land \lnot B\}}\tag{A.7}\\
	\notag
	\end{align}
	\item[\textbf{4.}] Schlussregeln:\\
	
\end{compactitem}
