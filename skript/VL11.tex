\section{Fortsetzungssemantik \tiny (Vorlesung 11 am 10.07.)}
% höhe Kovention
\marginnote{\small KLAUSUR am 17.7. 8-10 Uhr, Raum 005, kein Material!}[0,5cm]
\subsection{Problem}
Sprunganweisungen der Form $\underline{goto}\ L$ oder andere irreguläre Programmfortsetzungen.\\
\textbf{Idee:} $P = C_1;C_2;C_3;\dots;C_n$\\
Betrachte fehlerfreien Fall für $C_1;C_2$:\\
Bisher: $\mathcal{C}[\![ C_1;C_2 ]\!] z = \mathcal{C}[\![ C_2 ]\!](\mathcal{C}[\![ C_1 ]\!]) $\\

\subsection{Continuations (Fortsetzung)}
(Wadsworth, Strachey 1974)
%TODO Pfeile sollen hinter dem Semikollon stehen! 
%TODO Looppfeil bei C_n mit Beschriftung \lambdaz.z ist Fortsetzung von C_n
\begin{tikzpicture}[-,>=stealth',shorten >=1pt,auto,node distance=1cm,
  thick,main node/.style={font=\sffamily\Large\bfseries}]
  \node[main node] (1) {$C_1;$};
  \node[main node] (2) [right of=1] {$C_2;$};
  \node[main node] (3) [right of=2] {$\dots$};
  \node[main node] (4) [right of=3] {$\dots;$};
  \node[main node] (5) [right of=4] {$C_n$};
    \path[every node/.style={font=\sffamily\small}]
    (1) edge [bend right=60] node {Forts. v. $C_1$} (5)
    (2) edge [bend right=40] node {Forts. v. $C_2$} (5)
    ;
\end{tikzpicture}
\marginnote{\smallDie Fortsetzung von $C_1$ ist Zustandstransformation definiert durch die Ausführung der Reste $C_2;\dots;C_n$}[0,5cm]

Jetzt hat $\mathcal{C}[\![ C_i ]\!]c_i\ z$ Zugriff auf die zug. Fortsetzung.\\
Dann kann $\mathcal{C}[\![ \underline{goto}\ L ]\!]c\ z = C_L \ z$ kürzer $\mathcal{C}[\![ \underline{goto}\ L ]\!]c = C_L$, wobei $C_L$ die Fortsetzung ist, die bei der Marke $L$ beginnt.\\
Bei $\mathcal{C}[\![ C_1;C_2 ]\!] c = \mathcal{C}[\![ C_1 ]\!](\mathcal{C}[\![ C_2 ]\!] \ c) $\\

\subsection{Forsetzungssemantik von WHILE}
Syntaktische Bereihe und semantische Bereiche \lstinline!ZAHL, BOOL, KON, ZUSTAND, SPEICHER, EINGABE, AUSGABE, ...! wie zuvor.\\
\begin{align*}
\text{\lstinline!FORTSETZUNG!} &= \text{\lstinline!ZUSTAND!} \rightarrow (\text{\lstinline!ZUSTAND!} + {\underline{Fehler}_\perp}) \text{ für} \text{\lstinline!COM!}\\
\text{\lstinline!TERMFORT!} &= \text{\lstinline!ZAHL!} \rightarrow \text{\lstinline!ZUSTAND!} \rightarrow (\text{\lstinline!ZUSTAND!} + {\underline{Fehler}_\perp}) \text{ für} \text{\lstinline!TERM!}\\
&= \text{\lstinline!ZAHL!} \rightarrow \text{\lstinline!FORTSETZUNG!}\\
\text{\lstinline!BOOLFORT!} &= \text{\lstinline!BOOL!} \rightarrow \text{\lstinline!FORTSETZUNG!} \text{ für} \text{\lstinline!BT!}\\
\end{align*}
Schreibe die \underline{Typen} für die Semantikfunktionen auf.
\begin{align*}
\mathcal{C}: & \text{\lstinline!COM!} \rightarrow \text{\lstinline!FORTSETZUNG!} \rightarrow (\text{\lstinline!ZUSTAND!} + {\underline{Fehler}_\perp}) \\
\text{(\emph{kurz})} \mathcal{C}: & \text{\lstinline!COM!} \rightarrow \text{\lstinline!FORTSETZUNG!} \rightarrow \text{\lstinline!FORTSETZUNG!} \\
\mathcal{T}: & \text{\lstinline!TERM!} \rightarrow \text{\lstinline!TERMFORT!} \rightarrow \text{\lstinline!FORTSETZUNG!}\\
\mathcal{B}: & \text{\lstinline!BT!} \rightarrow \text{\lstinline!BOOLFORT!} \rightarrow \text{\lstinline!FORTSETZUNG!}\\
\end{align*}
semantische Klauseln:
\begin{align*}
\mathcal{T}[\![ n ]\!] k = & k \ n\\
\mathcal{T}[\![ x ]\!] k <s,e,a> = & sx = \underline{frei} \rightarrow \underline{Fehler}, k(sx)<s,e,a>\\
\mathcal{T}[\![ \underline{read} ]\!] k <s,e,a> = & e \neq <> \rightarrow (\underline{is}_{\text{\lstinline!ZAHL!}} (\underline{hd}\ e) <s, \underline{tl}\ e, a>), \underline{Fehler}\\
\mathcal{T}[\![ T_1 + T_2 ]\!] k = & \mathcal{T}[\![ T_1 ]\!] \lambda n_1. \mathcal{T}[\![ T_2 ]\!] \lambda	n_2 z.k(n_1 + n_2) | \text{Semantik über } \mathbb{Z}_\perp\\
= & \mathcal{T}[\![ T_1 ]\!] \lambda n_1. \mathcal{T}[\![ T_2 ]\!] \lambda	n_2 . \min > (n_1 + n_2) \lor \max < (n_1 + n_2) \rightarrow \underline{Fehler}, k (n_1+n_2) z\\
\\
\mathcal{C}[\![ \underline{skip} ]\!] = & \underline{id} \quad \text{alternativ} \quad \mathcal{C}[\![ \underline{skip} ]\!] c = c\\
\mathcal{C}[\![ I:=T ]\!] c = &\mathcal{T}[\![ T ]\!] \lambda n(s,e,a).c<s[n/I],e,a>\\
\mathcal{C}[\![ \underline{output}\ T ]\!] c = & \mathcal{T}[\![ T ]\!] \lambda n(s,e,a).c<s,e,a.n>\\
\mathcal{C}[\![ C_1;C_2 ]\!] c = & \mathcal{C}[\![ C_1 ]\!] \circ \mathcal{C}[\![ C_2 ]\!]\\
\mathcal{C}[\![ \underline{if}\ B \ \underline{then} \ C_1 \ \underline{else} \ C_2 ]\!]  = & \mathcal{B}[\![ B ]\!] (\underline{cond}<\mathcal{C}[\![ C_1 ]\!],\mathcal{C}[\![ C_2 ]\!]  >)\\
\mathcal{C}[\![ \underline{while}\ B \underline{do} \ C ]\!] = & \mathcal{B}[\![ B ]\!] (\underline{cond}<\mathcal{C}[\![ C ]\!] \circ \mathcal{C}[\![ \underline{while}\ B \underline{do} \ C ]\!], \underline{id}>)\\
\end{align*}
\subsection{Beispiel}
\begin{align*}
P = \underbrace{x = \underline{read};}_{C_1}\underbrace{\underline{if}\ x>0 \ \underline{then}\ \underbrace{\underline{output}\ 1}_{C_3}\ \underline{else}\ \underbrace{\underline{output}\ (-1)}_{C_4}}_{C_2}\\
\end{align*}
\begin{align*}
\mathcal{P}: & \text{\lstinline!COM!} \rightarrow \text{\lstinline!EINGABE!} \rightarrow (\text{\lstinline!AUSGABE!} + {\underline{Fehler}_\perp}) \\
\mathcal{P}[\![ P ]\!] e = & (\mathcal{C}[\![ P ]\!] (\lambda z.z) \bowtie \pi_3) <s_0,e, \epsilon>\\
\mathcal{C}[\![ P ]\!]\ c_0 \ z_0 = & \mathcal{C}[\![ C_1 ]\!] \underbrace{ (\mathcal{C}[\![ C_2 ]\!] c_0)}_{C_1}\ z_0 \quad \text{mit } z_0 = <s_0, <-24>,\epsilon>\\
=& \mathcal{T}[\![ \underline{read} ]\!] \underbrace{(\lambda n (s,e,a).c_1<s[n/x],e,a>)}_{k_1} z_0\\
=& k_1 (-24) <s_0,<>,\epsilon>\\
=& c_1 <s_0[-24/x,\epsilon,\epsilon]>\\
\vdots\\
\end{align*}