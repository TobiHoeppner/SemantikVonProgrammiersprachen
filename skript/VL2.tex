\section{Vorlesung 2 - 24.04.}
\subsection{Operationelle Symantik am Beispiel der Terme}
\emph{(Inhalt ist nicht im Lehrbuch!)}
%notes:
% auf höhe interpreter gfx!
\marginnote{\textbf{AST:} abstract syntaxtree}[1cm]

Es ist wichtig die Struktur von einer Sprache zu kennen, erst dann kann man eine korrekte Interpretation anfertigen!\\
Grundsätzlich gibt es zwei Methoden:
\begin{compactitem}
	\item[Übersetzer] Zu jedem Programm ein äquivalentes Maschinenprogramm erstellen.
	% gfx:übersetzer einfügen!
		
	\item[Interpreter]
	%gfx:interpreter einfügen!
\end{compactitem}
\subsubsection{Terme der Sprache WHILE}
\begin{lstlisting}
// Terme TERM
T::= Z  |  I  |  T1 OP T2 | read, für T1,T2 in TERM
\end{lstlisting}

\subsubsection{Beispiel: AST}
Der Ausdruck
\begin{lstlisting}
3 + read - x 
\end{lstlisting}
wird zu:\\
\begin{tikzpicture}[-,>=stealth',level/.style={sibling distance = 5cm/#1, level distance = 1.5cm}] 
\node [node]{-}
	child{node [node]{+}
		child{ node [node]{3}}
		child{ node [node]{read}}
	}
	child{node [node]{x}};
\end{tikzpicture}

\subsubsection{Informelle Semantik}
Interpretation nur möglich, wenn Speicher und Eingabe vorgelegt sind.\\
Annahme: wir bekommen alles als AST und wir bekommen eine Eingabe, die auch von der Maschine unterstützt wird!
\begin{compactitem}
	\item[Übersetzer] 
		\textbf{Idee:} depth-first-left-to-right-postorder Traversierung des AST
		Unser Beispiel:\\
		\begin{tikzpicture}[-,>=stealth',level/.style={sibling distance = 5cm/#1, level distance = 1.5cm}] 
		\node [node]{-}
			child{node [node]{+}
				child{ node [node]{3}}
				child{ node [node]{read}}
			}
			child{node [node]{x}};
		\end{tikzpicture}
		wird übersetzt zu:\\
	\begin{lstlisting}
PUSH 3
READ
ADD
LOAD x
SUB
		\end{lstlisting}
		Zustandsveränderungen:\\
		aktueller Zustand (siehe Bild Architektur!):
%e -> epsilon
<e|S|8.5. ...> -PUSH 3> <3.|S|8.5. ...> -READ> <-8.3-e|S|5. ...> -ADD> <3+(-8).e|S|5. ...> -LOAD x> <2.-5.e|S| ...5> -SUB> <-7.e|S|5...>

Semantik eines Terms T zu geg. Speicher S und Eingabe E ist die Spitze des Wertekellers(STACK) nach Ausführung von trans T auf <e|S|E>, falls diese Ausführung fehlerfrei läuft, sonst Fehler!\\
	\item[Interpreter] (abstrakte Maschine beinhaltet eine Komponente (Kontrollkeller), in der ASTs in einem Keller gespeichert werden können.)
	\begin{compactitem}
		\item Kontrollkeller
		\item Zustand der abstrakten Maschine hat Komponenten
			\begin{compactitem}
			%TODO mathe gedöns einfügen!
				\item Wertekeller W ($\in ZAHL*$)
				\item Speicher S ($S \in [ID \rightarrow Zahl]$)
				\item Kontrollkeller K ($ \in (AST \cup OP)*$)
				\item Eingabe E ($\in ZAHL*$)
			\end{compactitem}
	\end{compactitem}
		Zur Formalisierung der Semantik über die abstrakte Maschine mit dem Zustandsraum Z durch Angabe von:
		\begin{compactitem}
			\item[(i)] einem Anfangszustand $Z_{T,S,E}$ für jeden Term $T$, Speicher $S$ und Eingabe $E$.
			\item[(ii)] eine Zustandsüberführungsfunktion $\Delta : Z \rightarrow Z$ (partiell)
			\item[(iii)] Erklärung der Semantik über Iteration von $\Delta$		
		\end{compactitem}
		für Terme aus WHILE:\\
			\begin{compactitem}
			% e -> epsilon!
				\item[(i)]	$Z_{T_0,S_0,E_0} := <e|S_0|T_0.e|E_0>$
				\item[(ii)] $\Delta$ per Induktion über die Struktur der Kontrollkellerspitze\\
					$\Delta <W|S|n.K|E> = <n.W|S|K|E>$ für alle $n \in ZAHL, W,S,E$ wie oben.\\
					$\Delta <W|S|x.K|E> = <s(x).W|S|K|E>$ für alle $x \in ID$\\
					$\Delta <W|S|read.K|n.E> = <n.W|S|K|E>$ für alle $n \in ZAHL$\\
					$\Delta <W|S|T_1 OP T_2.K|E> = <W|S|T_1.T_2.OP.K|E>$ \\
					$\Delta <n_2.n_1.W|S|OP.K|E> = <n_1 OP n_2.W|S|K|E>$ $n_1, n_2 \in ZAHL$ falls $n_1 OP n_2$ definiert ist.\\			
				\item[(iii)] Die Semantik eines (beliebigen) Terms T im Bezug auf einem Speicher S und eine Eingabe E ist $n \in ZAHL$, wenn $\Delta^k Z_{T,S,E} = <n.e|S|e|E'>$ für beliebige $E' \in Zahl*$, undefiniert sonst!\\ 		
			\end{compactitem}		
\end{compactitem}

\subsubsection*{Architektur der abst. Maschine}
% gfx architektur!

\subsubsection*{Befehlssatz}
\begin{lstlisting}
// Stack und Speicher Operationen
READ 		// nimm Zahl von Eingabe, lege auf Stack, rücke Zeiger um eins weiter
LOAD x		// nimm Inhalt aus Speicher mit symbolischer Adresse x und lege auf Stack
PUSH n		// für jede Zahl aus N lege n auf Stack
(STORE x) // belegt Speicher mit der symbolischer Adresse x 
(GOTO n) 	// bedingter Sprung
// arithmetische Operationen
ADD
MULT
SUB
\end{lstlisting}




