\section{Vorlesung 5 - 22.05.}
\subsection{Denotationelle Semantik}
% auf hohe Notation:Funktionen
\marginnote{\small\emph{(Klammern sparen!)}}[4,5cm]
% höhe B.1.
\marginnote{\small\emph{(Abh. vom Speicher + Nebenwirkung)}}[8,5cm]
\textbf{Grundprinzip:} \\Direkte (injektive) Zuordnung von syntaktischen Strukturen zu deren Semantik (als mathematisches Objekt (Konstrukt)).
\subsubsection{Allgemeines Prinzip}
	\begin{compactitem}
		\item[\textbf{1.}] Ordne jeder syntaktischen Kategorie $K$ der Sprache einen semantischen Bereich $B_K$ zu.
		\item[\textbf{2.}] Schreibe (definiere) zu jeder syntaktischen Kategorie eine Semantikfunktion $\mathcal{K}$\\
		$\mathcal{K}:K \rightarrow B_K$
	\end{compactitem}
	\textbf{Notation:}
	\begin{compactitem}
		\item[-] Das Argument einer Semantikfunktion $\mathcal{K}$ schreiben wir in $
[\![$ und $]\!]$
		\item[-] Funktionen z.B. $f: A \rightarrow (B \rightarrow C)$ werden $f\;a\;b = a \dots b \dots $ 
	\end{compactitem}

\subsubsection{Denotationelle Semantik}
	\begin{compactitem}
		\item[-] \textbf{Syntaktische Kategorien} von WHILE: TERM, BT, COM, PROG
		\item[-] Hilfsbereiche:\\
		ZUSTAND = SPEICHER $\times$ EINGABE $\times$ AUSGABE mit SPEICHER = ID $\rightarrow$ ZAHL $\cup \{\underline{frei}\}$, \\EINGABE = KON$^*$ und AUSGABE = KON$^*$
		\item[\textbf{1.}] $\mathcal{T}:$TERM$\rightarrow \underline{\underline{B_T}}$\\
		$B_T :=$ZAHL???\\
		\fbox{\parbox[b]{\textwidth*4/5}{
		% versuchen als kleinen Einschub auf Höhe B.1. zu ordnen
			\underline{Exkurs: \textbf{Curry-Isomorphie}} \\
\begin{tikzpicture}[>=stealth',shorten >=1pt,auto,node distance=2.3cm]
  \node[state,draw=none] (q1)      {$(A \times B \rightarrow C)$};
  \node[state,draw=none] (q2) [right of=q1]  {$\cong$};
  \node[state,draw=none] (q3) [right of=q2] {$A \rightarrow (B \rightarrow C)$};
  \path[->] (q1)  edge   [bend left=30]   node {curry} (q3);
  \path[->] (q3)  edge   [bend left=30]   node {uncurry} (q1);
\end{tikzpicture}\\	
			%$(A \times B \rightarrow C) \cong A \rightarrow (B \rightarrow C)$\\
			\textbf{curry}: $f\;a\; b = f(a,b)$\\
			\textbf{uncurry}: $g\;(a,b) = g\;a\;b$\\
			$B_T := $ SPEICHER $\rightarrow$ EINGABE $\rightarrow$ ((ZAHL $\times$ Eingabe) $\cup \{\underline{Fehler} \}$ )\\
			$\mathcal{B}:$ BT $\rightarrow $ SPEICHER $\rightarrow$ EINGABE $\rightarrow$ ((BOOL $\times$ Eingabe) $\cup \{\underline{Fehler} \}$ )\\
			$\mathcal{C}:$ COM $\rightarrow $ $\underbrace{ \text{ZUSTAND} \rightarrow \text{ZUSTAND }\cup \{\underline{Fehler} \}\text{ )}}_{B_C}$\\
		% geschwungene Klammer mit $B_C$unter ZUSTAND -> (...)
		}}
		\item[\textbf{2.}] Induktive Definition über dem Aufbau
		% höhe \mathcal{C}[\underline{skip}] z &= z
\marginnote{\small\emph{alt. Schreibweise:}\\$\mathcal{C}
[\![\underline{skip}]\!]\;z = \underline{id}$}[11,5cm]
\marginnote{\small$z = (s,e,a)$}[12,5cm]
		\begin{align*}
		\mathcal{T}
[\![n]\!]\;s\;e &= (n,e) \quad\text{ für alle } n \in \text{ZAHL}\\
		\mathcal{T}
[\![x]\!]\;s\;e &= \left\lbrace\begin{array}{l l} (n,e)\text{,}\quad\text{falls } s\;x = n\\
		\underline{\text{Fehler}}\text{,}\quad\text{falls } s\;x = \underline{frei } \text{ für alle } x \in \text{ID} \end{array}\right.\\		
		\mathcal{T}
[\![\underline{read}]\!]\;s\;e &= \left\lbrace\begin{array}{l l} (n,e')\text{,}\quad\text{falls } e = n.e'\\
		\underline{\text{Fehler}}\text{,}\quad&\text{falls } e = \mathcal{E} \text{ oder } e = b.e' \text{ mit } b \in \text{BOOL} \end{array}\right.\\
		\mathcal{T}
[\![\underline{T_1 \; \underline{OP}\; T_2}]\!]\;s\;e &= \left\lbrace\begin{array}{l l} (n,e'')\text{,}\quad\text{falls } &\mathcal{T} 
[\![T_1]\!] s e = (n_1,e') \text{ und} \\  &\mathcal{T} 
[\![T_2]\!] s e' = (n_2,e'') \text{ und }\\ &\underline{n_1\;\underline{OP}\;n_2} = n \in \text{ZAHL}\\
		\underline{\text{Fehler}}\text{,}\quad\text{sonst} \end{array}\right.\\
		\\
		\mathcal{B}
[\![w]\!]\;s\;e &= (w,e) \quad\text{für alle} w \in \text{BOOL}\\
		\mathcal{B}
[\![\underline{not}B]\!]\;s\;e &=\left\lbrace\begin{array}{l l} (b,e')\text{,}\quad\text{falls } \mathcal{B}
[\![B]\!]s e = (w, e') \text{ und } b = \lnot w\\
		\underline{\text{Fehler}}\text{,}\quad\text{sonst}\end{array}\right.\\
		\mathcal{B}
[\![T_1\;\underline{BOP}\;T_2]\!]\;s\;e &=\left\lbrace\begin{array}{l l} (b,e'')\text{,}\quad\text{falls } &\mathcal{T} 
[\![T_1]\!] s e = (n_1,e') \text{ und} \\  &\mathcal{T} 
[\![T_2]\!] s e' = (n_2,e'') \text{ und } \\ &\underline{n_1\;\underline{BOP}\;n_2} = b \in \text{BOOL}\\
		\underline{\text{Fehler}}\text{,}\quad\text{sonst}\end{array}\right.\\
		\mathcal{B}
[\![\underline{read}]\!]\;s\;e &=\left\lbrace\begin{array}{l l} (b,e')\text{,}\quad\text{falls } e = b.e' \text{ mit } b \in \text{BOOL}\\
		\underline{\text{Fehler}}\text{,}\quad\text{falls } e = \mathcal{E} \text{ oder } e = n.e' \text{ mit } n \in \text{ZAHL}\end{array}\right.\\
		\\
		\mathcal{C}
[\![\underline{skip}]\!]\;z &= z\\
		\mathcal{C}
[\![I:=T]\!]\;(s,e,a) &= \left\lbrace\begin{array}{l l} (S[n/I], e', a)\text{,}\quad\text{falls }\mathcal{T}
[\![T]\!]\;s\;e = (n, e')\\
		\underline{\text{Fehler}}\text{,}\quad\text{sonst} \end{array}\right.\\
		\mathcal{C}
[\![\underline{output}\;T]\!]\;(s,e,a) &= \left\lbrace\begin{array}{l l} (s,e',a.n)\text{,}\quad\text{falls } \mathcal{T}
[\![T]\!]s e = (n,e')\\
		\underline{\text{Fehler}}\text{,}\quad\text{sonst} \end{array}\right.\\
		\mathcal{C}
[\![\underline{output}\;B]\!]\;(s,e,a) &= \left\lbrace\begin{array}{l l} (s,e',a.b)\text{,}\quad\text{falls } \mathcal{B}
[\![B]\!]s e = (b,e')\\
		\underline{\text{Fehler}}\text{,}\quad\text{sonst} \end{array}\right.\\
		\mathcal{C}
[\![C_1;C_2]\!]\;z &= \left\lbrace\begin{array}{l l} \mathcal{C} 
[\![C_2]\!] z', \mathcal{C} 
[\![C_1]\!] z = z' \quad\text{ mit } z' \in \text{ZUSTAND}\\
		\underline{\text{Fehler}}\text{,}\quad\text{sonst} \end{array}\right.\\
		\mathcal{C}
[\![\underline{if}\;B \underline{then}\;C_1\;\underline{else}\;C_2]\!]\;(s,e,a) &= \left\lbrace\begin{array}{l l} \mathcal{C} 
[\![C_1]\!]\;(s,e',a)\text{,}\quad\text{falls } &\mathcal{B}
[\![B]\!]\;s\;e = (\underline{true}, e')\\
		\mathcal{C}
[\![C_2]\!]\;(s,e',a)\text{,}\quad\text{falls } & \mathcal{B}
[\![B]\!]\;s\;e = (\underline{false}, e')\\
		\underline{\text{Fehler}}\text{,}\quad\text{sonst} \end{array}\right.\\
		\mathcal{C}
[\![\underline{while}\;B\;\underline{do}\;C]\!]\;(s,e,a) &= \left\lbrace\begin{array}{l l} \mathcal{C}
[\![C;\underline{while}\;B\;\underline{do}\;C]\!](s,e',a)\text{,}\quad\text{falls }&\mathcal{B}
[\![B]\!]\;s\;e = (\underline{true}, e')\\
		(s,e',a)\text{,}\quad\text{falls }&\mathcal{B}
[\![B]\!]\;s\;e = (\underline{false}, e')\\
		\underline{\text{Fehler}}\text{,}\quad\text{sonst} \end{array}\right.\\
		\\
		\mathcal{P}
[\![P]\!]\;e &= \left\lbrace\begin{array}{l l}a\text{,}\quad\text{falls }&\mathcal{C}
[\![C]\!](s_0,e,\epsilon) = (s, e', a)\\
		\text{Fehler,}\quad\text{falls } &\mathcal{C}
[\![C]\!](s_0,e,\epsilon) = \underline{Fehler}\\
		\text{undefiniert sonst} \end{array}\right.\\
		\end{align*}
\end{compactitem}