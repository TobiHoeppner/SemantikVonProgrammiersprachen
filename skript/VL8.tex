\section{ Konsturktion Semantischer Bereiche (cpo's) \tiny (Vorlesung 8 am 19.06.)}
% höhe von Grundlage
%\marginnote{\small\emph{()}}[0,5cm]
% höhe Allgemeine Methode 1.)
%\marginnote{\small\emph{()}}[3cm]
\subsection{Abzählbare Mengen}
Sei $M$ eine abzählbare Menge. Der cpo $ M_{\perp} $ entsteht aus $M$ durch
\begin{align*}
M_{\perp} = (M \cup \{ \perp \}, \sqsubseteq) \text{ mit } x \sqsubseteq y \text{ gdw. } x = \perp \text{ oder } x = y\\ &&\qed \\
\end{align*}
Beispiele:
% kleiner Baum neben Definition von ZAHL malen
% min, min+1, ... , 0, 1, ..., max
%		\			\		  |				/
%						ZAHL_\perp
% ebenfalls für N und Bool zeichnen
\begin{align*}
\text{ZAHL} = \{\min,...,\max\}_{\perp} \\
\end{align*}
\begin{figure}[H]
\centering
\begin{tikzpicture}[-,>=stealth',shorten >=1pt,auto,node distance=2cm,
  thick,main node/.style={font=\sffamily\Large\bfseries}]
  \node[main node] (1) {$\min ,$};
  \node[main node] (2) [right of=1] {$\min +1 ,$};
  \node[main node] (3) [right of=2] {$\dots ,$};
  \node[main node] (4) [right of=3] {$0,$};
  \node[main node] (5) [right of=4] {$1,$};
  \node[main node] (6) [right of=5] {$\dots ,$};
  \node[main node] (7) [right of=6] {$\max$};
  \node[main node] (8) [below of=4] {$\perp$};
  \path[every node/.style={font=\sffamily\small}]
    (1) edge node {} (8)
	(2) edge node {} (8)
	(4) edge node {} (8)
	(5) edge node {} (8)
	(7) edge node {} (8)
    ;
\end{tikzpicture}
\end{figure}

\begin{align*}
\mathbb{N}_{\perp} = 0, 1, 2, ... \\
\end{align*}

\begin{figure}[H]
\centering
\begin{tikzpicture}[-,>=stealth',shorten >=1pt,auto,node distance=2cm,
  thick,main node/.style={font=\sffamily\Large\bfseries}]
  \node[main node] (1) {$0,$};
  \node[main node] (2) [right of=1] {$1,$};
  \node[main node] (3) [right of=2] {$\dots ,$};
  \node[main node] (4) [right of=3] {$\infty$};
  \node[main node] (8) [below of=3] {$\perp$};
  \path[every node/.style={font=\sffamily\small}]
    (1) edge node {} (8)
    (2) edge node {} (8)
    (4) edge node {} (8)
    ;
\end{tikzpicture}
\end{figure}

\begin{align*}
\text{BOOL} = \{\underline{true}, \underline{false}\}_{\perp} \\
\end{align*}

\begin{figure}[H]
\centering
\begin{tikzpicture}[-,>=stealth',shorten >=1pt,auto,node distance=2cm,
  thick,main node/.style={font=\sffamily\Large\bfseries}]
  \node[main node] (1) {$\underline{true}$};
  \node[] (11) [right of=1] {};
  \node[main node] (2) [right of=11] {$\underline{false}$};
  \node[main node] (8) [below of=11] {$\perp$};
  \path[every node/.style={font=\sffamily\small}]
    (1) edge node {} (8)
    (2) edge node {} (8)
    ;
\end{tikzpicture}
\end{figure}
\begin{align*}
\text{ID} = \{ w | w \in \{a,b, \dots, z\}^* \}_{\perp} \\
\end{align*}
\subsection{Kartesische Produkt}
Seien $ D_1, ... , D_n $ cpo's. $ D = D_1 \times ... \times D_n $\\
$D$ ist cpo mit komponentenweise Ordnung $\sqsubseteq_D$, d.h.
\begin{align*}
<d_1,...,d_n> &\sqsubseteq_D <d'_1,...,d'_n> \text{ gdw. } d_1 \sqsubseteq_{D_1} d'_i \text{ für alle } 1 \leq i \leq n \\
D &:= (D_1 \times D_n, \sqsubseteq_D) \text{ ist cpo.} \\
\end{align*}
\subsection{Summen}
Seien: $ D_1, ... , D_n  $ cpo's\\
% Bild mit einem Baum neben für alle:
% D_1, D_2, ... D_n
% 	\			\			/
%					\perp_D
\begin{figure}[H]
\centering
\begin{tikzpicture}[-,>=stealth',shorten >=1pt,auto,node distance=2cm,
  thick,main node/.style={font=\sffamily\Large\bfseries}]
  \node[main node] (1) {$D_1,$};
  \node[main node] (2) [right of=1] {$D_2,$};
  \node[main node] (3) [right of=2] {$\dots ,$};
  \node[main node] (4) [right of=3] {$d_n$};
  \node[main node] (8) [below of=3] {$\perp$};
  \path[every node/.style={font=\sffamily\small}]
    (1) edge node {} (8)
    (2) edge node {} (8)
    (4) edge node {} (8)
    ;
\end{tikzpicture}
\end{figure}

\begin{align*}
D &= D_1 + ... + D_n \text{ ist wie folgt erklärt.}\\
D &= (\{(d,i) | d \in D_i, i \leq i \leq n\}, \sqsubseteq_D ) \cup \{\perp_D\}, \sqsubseteq_D) mit (d,i) \sqsubseteq_D (e,j) \text{ gdw. } i = j \text{ und } \perp_D \sqsubseteq (d,i) \text{ für alle ...}\\
\end{align*}
\subsection{endliche Folgen}
Sei $\underline{D}=(D,\sqsubseteq_D)$ ein cpo.
\begin{align*}
\underline{D^*} = (\{ <d_i | d_i \in 1 \leq i \leq n > | n \in \mathbb{N} \} \cup \{ \perp_{D^*}\}, \sqsubseteq_{D^*})\text{ mit}&\\
\text{\underline{Vorüberlegung}}& \\
& <d_i | 1 \leq i \leq n> \sqsubseteq_D <d'_i | 1 \leq i \leq m> \\
&\text{gdw. } n \leq m \text{ und } d_i \sqsubseteq_D d'_i \text{ für alle } 1 \leq i \leq n \\
& \text{ funktioniert nicht, weil}\\
&\text{Sei } 1 \in D,\\
& K:= \{ <>,<1>,<1,1>,<1,1,1>,... \} \text{ Kette in } D^* \\
&\bigsqcup K = <1,1,1,...> \text{ unendliche Folgen in } D^W\\
\text{Ende der Vorüberlegung}&\\
<d_i | 1 \leq i \leq n > \sqsubseteq_{D^*} <d'_i | 1 \leq i \leq n > &\\
\text{gdw. } n = m \text{ und } d_i \sqsubseteq_D d'_i \text{ für alle } 1 \leq i \leq n & \\
\perp_{D^*} \sqsubseteq_{D^*} \text{ für jede Folge } \in D^* &\\
\\
\text{d.h. } D^* = \bigcup_{i \in \mathbb{N}} D^i &\\
\end{align*}
\subsection{unendliche Folgen}
\begin{align*}
D^w := (\{ < d_i | i \in \mathbb{N} > \}, \sqsubseteq_{D^W}) \text{ mit } <d_i | i \in \mathbb{N} > \sqsubseteq_{D^w} <d'_i | i \in \mathbb{N}> \text{ gdw. } d_i \sqsubseteq_{D} d'_i \text{ für alle } i \in \mathbb{N}\\
\perp_{D^W} = <\perp_D | i \in \mathbb{N}>\\
\end{align*}
\subsection{Funktionenraum}
Seien $D_1$ und $D_2$ cpo's
\begin{align*}
D = [D_1 \rightarrow D_2] &:= (\{ f : D_1 \rightarrow D_2 | f \text{ ist stetig}\}, \sqsubseteq_{D}) \text{ mit punktweiser Ordnung, d.h.}\\
f &\sqsubseteq_{D} g \text{ gdw. } f(d) \sqsubseteq_{D_2} g(d) \text{ für alle } d \in D_1 \text{.} \\
\end{align*}
$D$ ist cpo!\\
\subsection{Ausblick}
Konstruktionen rekrusiver Domains ist möglich. Wenn $D$ eine cpo-Variable, dann löst sich eine Gleichung der Form $D = \tau [D, B_1,...,B_r]$ lösen. $\tau$ über $\times, +, *, ^w, \rightarrow $, cpo's $B_i$\\ 
Kanonische Operationen, wie Projektionen, Injektionen, Test auf leere Liste, ..., sind stetig.\\
